
%\documentclass[xcolor=dvipsnames,10pt]{beamer}
\documentclass[xcolor=dvipsnames,10pt,handout, draft]{beamer}

\usetheme{Madrid}


\definecolor{coulourname}{rgb}{.25,.5,.25}

\usecolortheme[named=coulourname]{structure}


%\usepackage[spanish]{babel}


\DeclareMathOperator{\gene}{span}
\DeclareMathOperator{\abs}{abs}
\DeclareMathOperator{\sen}{sen}

\usepackage{import}

\newcommand{\incfig}[1]{%
    \def\svgwidth{3.5in}
    \import{./figuras/}{#1.pdf_tex}
}

\newcommand{\nz}{\mathbb N}
\newcommand{\rz}{\mathbb R}
\newcommand{\rzp}{\mathbb R^{p}}
\newcommand{\rzq}{\mathbb R^{q}}
\newcommand{\rzm}{\mathbb R^{m}}
\newcommand{\qzp}{\mathbb Q^{p}}
\newcommand{\intunit}{\mathbb I}


\newtheorem*{afirma}{}
\newtheorem*{vgr}{Ejemplo}
\newtheorem*{teo}{Teorema}
\newtheorem*{propi}{Propiedades}
\newtheorem*{propo}{Proposici\'on}
\newtheorem*{cor}{Corolario}

\author[MAT24111]{An\'alisis Matem\'atico 2\\Prof. J. Rivera Noriega}
\institute{ITAM}

\title{Repaso Primer Parcial}

\date{Otoño de 2020}

\begin{document}


\frame{\titlepage}


\begin{frame}{Otro modo de ver continuidad puntual}

Recordemos la definici\'on de continuidad de una funci\'on en un punto que funciona en cualquier espacio m\'etrico $(X,d)$. \pause\\
Sean $f:D(f)\subseteq X\to X$ y $a\in D(f)$. \pause

\begin{afirma}
$f$ es continua en $a\in D(f)\subseteq X$ si para toda $\epsilon>0$ existe $\delta>0$ tal que 
$$x\in D(f)\cap B_\delta(a)\quad\text{implica}\quad f(x)\in B_\epsilon(f(a))$$
\end{afirma}\pause
Esta definici\'on puede escribirse usando im\'agenes inversas:\pause
\begin{afirma}
$f$ es continua en $a$ si para toda $\epsilon>0$ existe $\delta>0$ tal que 
$$f^{-1}\big(B_\epsilon(f(a))\big)\supseteq D(f)\cap B_\delta(a).$$
\end{afirma}\pause
Este \'ultimo punto de vista tiene la ventaja de usar propiedades de abiertos en el dominio y en la imagen.

\end{frame}


\begin{frame}

\begin{figure}[ht]
    \centering
    % \incfig{ContinuidadPuntual}
    \caption{\em $x\in D(f)\cap B_\delta(a)$ implica que $f(x)\in B_\epsilon(f(a))$}
\end{figure}

\end{frame}


\begin{frame}{Continuidad global}

La observaci\'on anterior es \'util cuando consideramos la continuidad de la funci\'on {\color{red!77!black} en todo punto} de $D(f)$.
\vskip4pt\pause
Se dice que $f$ es {\color{red!77!black} continua en su dominio} si es continua en todo punto de $D(f)$.\pause

\begin{teo}[de continuidad global - Versi\'on 1]
Para $f:D(f)\subseteq\rzp\to\rzq$ son equivalentes las siguientes condiciones:
\begin{itemize}
\item[(a)] $f$ es continua en su dominio;\pause
\item[(b)] Si $G\subseteq\rzq$ es abierto, existe $G_1\subseteq\rzp$ abierto tal que $\displaystyle f^{-1}(G)=D(f)\cap G_1$;\pause
\item[(c)] Si $H\subseteq\rzq$ es cerrado, existe $H_1\subseteq\rzp$ cerrado tal que $\displaystyle f^{-1}(H)=D(f)\cap H_1$;\pause
\end{itemize}
\end{teo}
Existe tambi\'en una versi\'on en la que el dominio de $f$ es todo $\rzp$.

\end{frame}


\begin{frame}{Continuidad global}

\begin{teo}[de continuidad global - Versi\'on 2]
Para $f:\rzp\to\rzq$ son equivalentes las siguientes condiciones:
\begin{itemize}
\item $f$ es continua en $\rzp$;\pause
\item Si $G\subseteq\rzq$ es abierto entonces $\displaystyle f^{-1}(G)$ es abierto en $\rzp$;\pause
\item Si $H\subseteq\rzq$ es cerrado entonces $\displaystyle f^{-1}(H)$ es cerrado en $\rzp$;\pause
\end{itemize}
\end{teo}

Debe observarse que este teorema es consecuencia inmediata del anterior, por lo que usualmente se demuestra la {\em Versi\'on 1} del teorema.
\vskip4pt\pause
La demostraci\'on est\'a en [Bartle, Section 22]
\end{frame}



\begin{frame}{Demostraci\'on del Teorema de Continuidad Global}

{\color{green!59!black} $\text{(a)}\,\Rightarrow\,\text{(b)}\quad$} Sea $G\subseteq\rzq$ abierto y $a\in f^{-1}(G)$. \pause\vskip4pt Como $G$ es {\em vecindad} de $f(a)$, sabemos que existe $\epsilon>0$ tal que $B_\epsilon(f(a))\subseteq G$, \pause\vskip4pt Por continuidad de $f$ en $a$ existe $\delta_a>0$ tal que 
$$f^{-1}(G)\supseteq f^{-1}\big(B_\epsilon(f(a))\big)\supseteq B_{\delta_a}(a)\cap D(f) \pause$$
Denotemos por $U_a=B_{\delta_a}(a)$ y repitamos el procedimiento para toda $a\in f^{-1}(G)$. \pause\vskip4pt Definimos $\displaystyle G_1=\bigcup_{a\in f^{-1}(G)}U_a$ que es abierto y que cumple $f^{-1}(G)=G_1\cap D(f)$:
\pause\vskip8pt
{\color{green!59!black} $\subseteq\quad$} es directo por construcci\'on.\pause\vskip4pt
{\color{green!59!black} $\supseteq\quad$} Si $x\in G_1\cap D(f)$ entonces $x\in U_{a_0}$ para alg\'un $a_0\in f^{-1}(G)$. \pause Pero como se vi\'o antes tendr\'\i amos
$$x\in U_{a_0}\cap D(f)\pause\subseteq f^{-1}(G)$$
\hfill$\square$

\end{frame}


\begin{frame}

\begin{figure}[ht]
    \centering
    % \incfig{ContinuidadGlobal}
    \caption{\em El procedimiento anterior se repite para toda $a \in f^{-1}(G)$}
\end{figure}

\end{frame}


\begin{frame}{Demostraci\'on del Teorema de Continuidad Global}

{\color{green!59!black} $\text{(b)}\,\Rightarrow\,\text{(a)}\quad$} 
Sean $a\in D(f)$ y $\epsilon>0$, de manera que con $G=B_\epsilon(f(a))\subseteq\rzq$ abierto, por hip\'otesis existe $G_1\subseteq\rzp$ abierto tal que $\displaystyle f^{-1}(G)=D(f)\cap G_1$.
\vskip4pt\pause
\'Esto implica que $a\in G_1$, por lo que existe $\delta>0$ tal que $B_\delta(a)\subseteq G_1$.
\vskip4pt\pause
Pero \'esto significa que 
$\displaystyle f^{-1}\big(B_\epsilon(f(a)\big)=f^{-1}(G)\pause=D(f)\cap G_1\pause\supseteq D(f)\cap B_\delta(a)$\hfill$\square$
\vskip8pt\pause
Se deja de ejercicio probar que (b) es equivalente con (c).\hfill$\blacksquare$

\vskip6pt\pause
N\'otese que este teorema se refiere a {\em imágenes inversas} de abiertos y cerrados. \pause\vskip4pt El siguiente ejemplo muestra que las {\em im\'agenes directas} de abiertos bajo funciones continuas no siempre dan lugar a un abierto:\pause
$$f(x)=\frac{1}{1+x^2},\qquad\pause G=(-1,1),\qquad\pause f(G)=(1/2,1]\pause$$ 
Cabe entonces preguntar cuáles propiedades son preservadas a trav\'es de imagen directa de funciones continuas.

\end{frame}



\begin{frame}{Preservaci\'on de la conexidad}

\begin{teo}
Sean $f:D(f)\subseteq\rzp\to\rzq$ y $H\subseteq D(f)$ conexo en $\rzp$. Si $f$ es continua en $H$, entonces $f(H)$ es conexo en $\rzq$.
\end{teo}\pause

La demostraci\'on completa est\'a en [Bartle, Section 22]. \pause
En esa prueba se considera la restricci\'on de $f$ a $H$, que denotamos por $h=f\big|_H$ es decir que $D(h)=H$ y $h(x)=f(x)$.
\vskip4pt\pause
N\'otese que $f(H)=h(H)$ y $h$ es continua en $H$.
\vskip4pt\pause
La prueba luego establece que si $h(H)$ fuera disconexo en $\rzq$, existir\'\i a $(A,B)$ disconexi\'on de $H$. 

\end{frame}


\begin{frame}{Teorema del valor intermedio}

Como aplicaci\'on importante del teorema anterior podemos establecer una propiedad fundamental de las funciones continuas con valores en $\rz$.\pause

\begin{teo}[Bolzano]
Sean $f:D(f)\subseteq\rzp\to\rz$ y $H\subseteq D(f)$ conexo en $\rzp$, y sup\'ongase que $f$ es continua y acotada en $H$.\\ \pause
Si $k\in\rz$ cumple $\displaystyle\inf\big\{f(x):x\in H\big\}<k<\sup\big\{f(x):x\in H\big\}$\pause\\
entonces existe $x\in H$ tal que $f(x)=k$, \pause (es decir $k\in f(H)$).\pause
\end{teo}\pause

Este resultado junto con otros que a continuaci\'on recordaremos establecen importantes propiedades de funciones $f:[a,b]\to\mathbb R$ continuas en todo $[a,b]$, intervalo cerrado y acotado. \pause\vskip4pt En alg\'un texto importante de C\'alculo se llaman los {\em ``tres teoremas fuertes''}.

\end{frame}


\begin{frame}{Preservaci\'on de la compacidad}\pause

\begin{teo}
Sean $f:D(f)\subseteq\rzp\to\rzq$ y $K\subseteq D(f)$ compacto en $\rzp$, y sup\'ongase que $f$ es continua en $K$. Entonces $f(K)$ es compacto en $\rzq$. \pause
\end{teo}

Usando la idea de la restricci\'on de $f$ al conjunto $K$, como se hizo antes, podemos suponer que $D(f)=K$.
\vskip4pt\pause
Sea $\mathcal G=\big\{G_\alpha:\alpha\in\mathcal A\big\}$ una cubierta abierta de $f(K)$.
\vskip4pt\pause
Por el teorema de continuidad global sabemos que existen $C_\alpha\subseteq\rzp$ abiertos tales que $f^{-1}(G_\alpha)=C_\alpha\cap K$.
\vskip4pt\pause
N\'otese que $\mathcal C=\big\{C_\alpha:\alpha\in\mathcal A\big\}$ es cubierta de $K$:\pause\\
{\color{green!37!black}\em Dado $x\in K$ se tendr\'a $\displaystyle f(x)\in f(K)$, \pause o sea que $x\in G_{\alpha_0}$, y entonces $x\in C_{\alpha_0}$.}\hfill$\square$
\vskip4pt\pause
Por ser $K$ compacto tendremos $K\subseteq C_{\alpha_1}\cup\cdots\cup C_{\alpha_N}$, \pause lo cual implica que $\displaystyle f(K)\subseteq G_{\alpha_1}\cup\cdots\cup G_{\alpha_N}$.

\end{frame}



\begin{frame}{Teorema del m\'aximo y el m\'\i nimo}

El teorema anterior nos permite establecer otra muy importante propiedad de funciones continuas con valores en $\rz$.\pause

\begin{teo}
Sean $f:D(f)\subseteq\rzp\to\rz$ y $K\subseteq D(f)$ compacto en $\rzp$, y sup\'ongase que $f$ es continua en $K$. Entonces existen $x^*,x_*\in K$ tales que 
$$f(x^*)=\sup\big\{f(x):x\in K\big\},\qquad f(x_*)=\inf\big\{f(x):x\in K\big\}$$
\end{teo}\pause
N\'otese primero que por el teorema anterior $f(K)$ es compacto en $\rz$, por tanto acotado.
\vskip4pt\pause
Sabemos pues de la existencia de $M=\sup f(K)$, \pause\vskip4pt Por la propiedad del supremo podemos construir $(x_n)$ sucesi\'on en $K$ tal que $\displaystyle f(x_n)>M-\frac{1}{n}$ para toda $n\in\mathbb N$. \pause O sea $\displaystyle M-f(x_n)<\frac{1}{n}$.

\end{frame}



\begin{frame}{Teorema del m\'aximo y el m\'\i nimo - 2}

Por el Teorema de Bolzano-Weierstrass existe una subsucesi\'on $(x_n')$ que converge a cierto $x^*\in K$.
\vskip4pt\pause
Al evaluar en este punto y usar que $f$ es coninua en $K$ obtendremos
$$f(x^*)=\lim f(x_n')=M\pause$$
Una prueba similar funciona para hallar $x_*\in K$ cumpliendo 
$$f(x_*)=\lim f(y_n')=m:=\inf f(K)\pause$$
\begin{teo}
\pause Sean $f:D(f)\subseteq\rzp\to\rzq$ y $K\subseteq D(f)$ compacto en $\rzp$, y sup\'ongase que $f$ es continua en $K$. Entonces existen $x^*,x_*\in K$ tales que 
$$\|f(x^*)\|=\sup\big\{\|f(x)\|:x\in K\big\},\qquad \|f(x_*)\|=\inf\big\{\|f(x)\|:x\in K\big\}$$
\end{teo}

\end{frame}



\begin{frame}{Espacios de funciones continuas y funciones acotadas}\pause

Fijando ahora $D\subseteq\rzp$, definimos
\begin{eqnarray*}
C_{pq}(D)&:=&\big\{f:D\to\rzq\,\big|\,f\text{ es continua en }D\big\}\pause\\
BC_{pq}(D)&:=&\big\{f:D\to\rzq\,\big|\,f\text{ es continua y acotada en }D\big\}\pause
\end{eqnarray*}
No es dif\'\i cil verificar que $C_{pq}(D)$ y $BC_{pq}(D)$ son espacios vectoriales bajo las operaciones usuales:
$$(f+g)(x)=f(x)+g(x),\quad (cf)(x)=cf(x),\qquad\text{para }\,x\in D\pause$$
Adem\'as $BC_{pq}(D)$ es un espacio normado con la norma
$$\|f\|_{\infty,D}:=\sup\big\{\|f(x)\|:x\in D\big\}\pause$$
Finalmente n\'otese que si $D$ es compacto entonces $C_{pq}(D)=BC_{pq}(D)$.
\end{frame}



\begin{frame}{Continuidad uniforme}

Dada $f:D(f)\subseteq\rzp\to\rzq$ y $A\subseteq D(f)$, se dice que $f$ es uniformemente continua en $A$ si para toda $\epsilon>0$ existe $\delta>0$  tal que para todo $x,u\in A$ que cumpla $\|x-u\|<\delta$ se tendr\'a $\|f(x)-f(u)\|<\epsilon$.
\vskip4pt\pause
N\'otese que si $f$ es uniformemente continua en $A\subseteq D(f)$ entonces continua en todo $a\in A$. 
\vskip4pt\pause
Sin embargo el rec\'\i proco es falso:\pause\\
Basta considerar la funci\'on $\displaystyle g(x)=\frac{1}{x}$ para $x>0$.\pause\vskip4pt
Intuitivamente, dada la misma $\epsilon>0$, entre m\'as cerca estamos de $x=0$, se va requiriendo una $\delta>0$ m\'as peque\~na. \pause \'Esto puede verse al tratar de estimar
$$|g(x)-g(u)|=\frac{|u-x|}{ux}\leq\frac{\delta}{ux}\qquad\text{suponiendo }\, |x-u|<\delta\pause$$
Al intentar que $\displaystyle \frac{\delta}{ux}<\epsilon$, entre m\'as peque\~nas son $x$ y $u$ se requerir\'a $\delta<\epsilon (ux)$ m\'as peque\~na.\hskip2cm{\color{blue!77!black}(ver detalles en [Bartle, pag. 159])}

\end{frame}



\begin{frame}{Criterio para la NO continuidad uniforme}

Mirando a la definici\'on de continuidad uniforme, y obteniendo la ``negaci\'on'' de dicha definici\'on, se puede justificar la siguiente afirmaci\'on:\pause\vskip4pt
\begin{itemize}
\item Para verificar que $f:D(f)\subseteq\rzp\to\rzq$ NO es uniformemente continua en $A\subseteq D(f)$ basta exhibir una $\varepsilon_0>0$ y dos sucesiones $(x_n)$ y $(y_n)$ en $A$ tales que, aunque $\|x_n-y_n\|<1/n$, se cumplir\'a $\|f(x_n)-f(y_n)\|\geq\varepsilon_0$.\pause
\end{itemize}
Por ejemplo, en el caso anterior, si $\varepsilon_0=1/2$ y elegimos $\displaystyle x_n=\frac{1}{n}$, $\displaystyle y_n=\frac{1}{2n}$. \pause tendremos $\displaystyle|x_n-y_n|=\frac{1}{2n}<\frac{1}{n}$ \pause pero 
$$|f(x_n)-f(y_n)|=\left|\frac{1}{x_n}-\frac{1}{y_n}\right|=n>\varepsilon_0$$

\end{frame}



\begin{frame}{Teorema de la continuidad uniforme}

\begin{teo}
Sea $f:D(f)\subseteq\rzp\to\rzq$ continua en su dominio. Si $K\subseteq D(f)$ es compacto entonces $f$ es uniformemente continua en $K$ 
\end{teo}

En [Bartle, pag. 160] hay dos demostraciones. 

\pause\vskip4pt
Una clase de ejemplos de funciones uniformemente continuas en su dominio son aquellas para las que existe $M>0$ tal que
$$\|f(x)-f(y)\|\leq M\|x-y\|\qquad\text{para toda }\,x,y\in D(f)\pause$$
A tales funciones se les llaman {\color{red!77!black}funciones de tipo Lipschitz}.
\vskip4pt\pause
Y dentro de esta clase ser\'an de nuestro inter\'es las {\color{red!77!black}contracciones}, es decir aquellas para las que $0<M<1$.
\vskip4pt\pause
En espacios m\'etricos esto quiere decir que $f:X\to X$ cumple
$$d(f(x),f(y))\leq Md(x,y)\qquad\text{para toda }\,x,y\in X,\quad\text{con }\,0<M<1.$$
\end{frame}



\begin{frame}{Funciones Lipschitz}
Una observaci\'on importante es que {\color{blue!77!black} no toda funci\'on uniformemente continua es de tipo Lipschitz}. \pause\vskip4pt Por ejemplo $f(x)=\sqrt x$ con $D(f)=[0,1]$ es uniformemente continua en su dominio, \pause pero no es de tipo Lipschitz, pues con $y=0$ se tendr\'\i a $\sqrt x\leq M x$ para $x\in(0,1]$ y cierta $M>0$; \pause o sea que $x\geq1/M^2$ para todo $x\in(0,1]$, lo cual es falso.
\end{frame}


\begin{frame}{Teorema de punto fijo para contracciones en espacios m\'etricos}
Sea $f:X\to X$, donde $(X;d)$ es un espacio m\'etrico. 
\vskip4pt\pause
Se dice que $u\in X$ es un {\color{red!77!black} punto fijo} de $f$ si $f(u)=u$.
\pause\vskip4pt
Similarmente a como se hizo en dominios de $\mathbb R^n$, veremos que las contracciones tienen puntos fijos, cuando el espacio m\'etrico donde se trabaja es completo \pause (toda sucesi\'on de Cauchy es convergente).
\begin{teo}
Sean $(X;d)$ un espacio m\'etrico completo y $f:X\to X$ una contracci\'on. Entonces existe $u\in X$ que es punto fijo de $f$.\pause
\end{teo}

Para la demostraci\'on construiremos una sucesi\'on contractiva, cuyo l\'\i mite ser\'a el punto fijo.
\vskip4pt\pause
Iniciamos con $x_1\in X$ arbitrario, y recursivamente $x_{n+1}=f(x_n)$ para $n\in\mathbb N$.
\vskip4pt\pause
Por la propiedad de contracci\'on se cumplir\'a:
$$d(x_{n+1},x_n)=d(f(x_n),f(x_{n-1}))\leq M d(x_n,x_{n-1}),\qquad0<M<1.$$

\end{frame}


\begin{frame}{Teorema del punto fijo para contracciones}
Recordemos ahora que las sucesiones contractivas como $(x_n)$ son sucesiones de Cauchy en $X$, que por la hip\'otesis ser\'an convergentes, con l\'\i mite en $X$.  
\vskip6pt\pause
Primero observemos que $d(x_k,x_{k-1})\leq Md(x_{k-1},x_{k-2})\pause\leq M^{k-2}d(x_2,x_1)$.
\pause\vskip4pt
Dados $m>n$, usando varias veces la desigualdad triangular, escribimos
\begin{align*}
d(x_m,x_n)&\leq d(x_m,x_{m-1})+d(x_{m-1},x_{m-2})+\cdots+d(x_{n+1},x_{n})\\
&\leq d(x_2,x_1)\big[M^{m-2}+M^{m-3}+\cdots+M^{n-1}\big]\\
&\leq d(x_2,x_1)M^{n-1}\big[M^{m-n-1}+M^{m-n-2}+\cdots+1\big]\\
&<\frac{M^{n-1}}{1-M}d(x_2,x_1)
\end{align*}
\pause
Y \'esto lleva a que $(x_n)$ es una sucesi\'on de Cauchy en $X$.
\end{frame}


\begin{frame}{Teorema del punto fijo para contracciones}

Como $X$ es completo, la sucesi\'on es convergente, digamos que $\lim x_n=u$.
Ahora veremos que $u$ es punto fijo de $f$, sustituyendo en la f\'ormula recursiva para obtener $u=f(u)$.
\vskip4pt\pause
Veremos ahora que adem\'as dicho punto fijo es \'unico: \pause Si $u,v\in X$ cumplen $u=f(u)$, $v=f(v)$ entonces 
$$d(u,v)=d(f(u),f(v))\leq Md(u,v)\pause$$
Si ocurriera que $u\neq v$ obtendr\'\i amos $1\leq M$ (!!)\hfill$\blacksquare$
\vskip4pt\pause
Recurriendo a los c\'alculos anteriores de hecho se puede calcular qu\'e tan cerca est\'a el t\'ermino $n$-\'esimo del l\'\i imite: 
$$d(u,x_n)\leq\frac{M^{n-1}}{1-M}d(x_2,x_1).$$

\end{frame}


\begin{frame}{Tema principal de la primera parte del curso}
Como el \'ultimo teorema muestra existe la posibilidad te\'orica de considerar conceptos topol\'ogicos en espacios de funciones tales como
\begin{eqnarray*}
C_{pq}(D)&:=&\big\{f:D\to\rzq\,\big|\,f\text{ es continua en }D\big\}\pause\\
BC_{pq}(D)&:=&\big\{f:D\to\rzq\,\big|\,f\text{ es continua y acotada en }D\big\}\pause
\end{eqnarray*}
De particular inter\'es se puede considerar el caso en que $D$ es compacto, en cuyo caso $C_{pq}(D)=BC_{pq}(D)$, donde de hecho tenemos la norma
$$\|f\|_{\infty,D}:=\sup\big\{\|f(x)\|:x\in D\big\}\pause$$
\pause
Surge entonces la necesidad de estudiar ahora sucesiones de funciones y sus distintos modos de convergencia, que provienen de la posibilidad de asignarle al mismo espacio  distintas m\'etricas.
\end{frame}



%%%%%%%%%%%%%%%%%%%%%%%%%%%%%%%%%%%%%%%%%%%%%%%%%%%%%%%%%%%
%%%%%%%%%%%%%%%%%%%%%%%%%%%%%%%%%%%%%%%%%%%%%%%%%%%%%%%%%%%
%%%%%%%%%%%%%%%%%%%%%\end{document}%%%%%%%%%%%%%%%%%%%%%%%%%%%%%%
%%%%%%%%%%%%%%%%%%%%%%%%%%%%%%%%%%%%%%%%%%%%%%%%%%%%%%%%%%%
%%%%%%%%%%%%%%%%%%%%%%%%%%%%%%%%%%%%%%%%%%%%%%%%%%%%%%%%%%%



\begin{frame}{Sucesiones de funciones -- Convergencia puntual}
Al considerar ahora la posibilidad de estudiar topolog\'\i a en el espacio $C(K)$ de funciones continuas definidas en un compacto $K$ dentro de un espacio m\'etrico, un primer paso es estudiar \textcolor{red}{sucesiones de funciones} y sus posibles \textcolor{red}{modos de converencia}.
\pause\vskip2pt
Fijando un dominio $D\subseteq\mathbb R^p$ se consideran funciones $f_n:D\to\mathbb R^q$ para cada $n\in\mathbb N$, para as\'\i\ formar la sucesi\'on $(f_n)$.
\pause\vskip2pt
La sucesi\'on \textcolor{red}{converge puntualmente} en $D$ a cierta funci\'on $f:D\to\mathbb R^q$ si para toda $x\in D$ la sucesi\'on en $\rzq$ definida por $(f_n(x))$ converge a $f(x)$.
\pause\vskip2pt
Para abreviar, en este caso escribiremos $f=\lim f_n$ o bien $f_n\to f$ puntualmente. 
\end{frame}

\begin{frame}{Ejemplos}
\begin{itemize}
\item $\displaystyle f_n(x)=\frac{x}{n}$, $D=\rz$, converge a la funci\'on constante $f\equiv0$ puntualmente\pause
\item $p_n(x)=x^n$, $D=[0,1]$, converge puntualmente a la funci\'on
$$\varphi(x)=\left\{\begin{array}{ll}
				0&\text{si }\,x\in[0,1)\\
				1&\text{si }\,x=1
			\end{array}\right.$$
\end{itemize}
\begin{figure}
\includegraphics[ height=4.5cm, width=4.5cm]{Potencias.pdf}
\caption{Gr\'aficas de $y=x^n$, con $n=1,2, 3, 5$  y $9$}
\end{figure}
\end{frame}

\begin{frame}{Ejemplos}
\begin{itemize}
\item $\displaystyle g_n(x)=\frac{x^2+nx}{n}$, $D=\rz$, converge a $g(x)=x$, pues de hecho se puede escribir $\displaystyle g_n(x)=\frac{x^2}{n}+x$.\pause
\item $\displaystyle h_n(x)=\frac{1}{n}\sen(nx+n)$, $D=\rz$, converge a $f\equiv0$. \pause Para justificar \'esto notemos que para cualquier $x\in\rz$
$$|h_n(x)-f(x)|=\frac{1}{n}|\sen(nx+n)|\leq\frac{1}{n}\pause$$
y como esta sucesi\'on tiende a cero, concluimos.\pause
\end{itemize}
\vskip6pt
\textcolor{blue}{Recu\'erdese: si se escribe $|x_n-L|\leq Ca_n$ con $a_n\to0$ y $C>0$ constante, entonces $x_n\to L$.}
\end{frame}

\begin{frame}{Convergencia uniforme}
Cuando escribimos la definici\'on rigurosa de convergencia puntual hallamos algo interesante: la sucesi\'on de funciones $(f_n)$, con $f_n:D\subseteq\rzp\to\rzq$, converge puntualmente a la funci\'on $f:D\subseteq\rzp\to\rzq$ en $D$ si\pause
$$\forall\epsilon>0,\;\textcolor{red}{\forall x\in D},\; \exists k=k(\epsilon,\textcolor{red}{x})\in\nz\;\text{ tal que }\;n\geq k\Rightarrow\;\|f_n(x)-f(x)\|<\epsilon$$\pause
Si cambiamos un poco los cuantificadores...
$$\forall\epsilon>0,\;\exists k=k(\epsilon)\in\nz\;\text{ tal que }\;n\geq k\Rightarrow\;\|f_n(x)-f(x)\|<\epsilon\quad\textcolor{red}{\forall x\in D},\; $$\pause
obtenemos la definici\'on de convergencia uniforme.
\vskip4pt\pause
Veamos la diferencia esencial: \pause

En la convergencia puntual, el \'\i ndice $k\in\nz$ que existe para cada $\epsilon>0$ \textcolor{blue}{puede cambiar} para distintas $x\in D$.\pause

En la convergencia uniforme, el \'\i ndice $k\in\nz$ que existe para cada $\epsilon>0$ \textcolor{blue}{es el mismo} para distintas $x\in D$.
\end{frame}

\begin{frame}{Convergencia uniforme}
Hay adem\'as un modo gr\'afico de entender la convergencia uniforme. \pause\vskip4pt
Esencialmente hay que dibujar una {\em vecindad tubular} alrededor de la funci\'on l\'\i mite, y tendr\'\i amos que ver que a partir de alg\'un \'\i ndice $k\in\nz$ todas las gr\'aficas de $f_n$, $n\geq k$, est\'an en esa {\em vecindad tubular}\pause
\begin{figure}[ht]
    \centering
    % \incfig{Tubular}
\caption{Representaci\'on gr\'afica de la convergencia uniforme}
\end{figure}
\end{frame}


\begin{frame}{Ejemplos}
De los ejemplos anteriores podemos extraer algunas ideas.\pause
\begin{itemize}
\item La sucesi\'on $\displaystyle h_n(x)=\frac{1}{n}\sen(nx+n)$ \textcolor{red}{converge uniformemente} porque se logr\'o obtener $|h_n(x)-f(x)|\leq 1/n$, y en el lado derecho no hay dependencia de $x$.\pause
\item La sucesi\'on $\displaystyle f_n(x)=\frac{x}{n}$ \textcolor{red}{no converge uniformemente} en $\rz$, \pause pues eligiendo $x_k=k$ tendremos $f_k(x_k)=1$ y por tanto $|f_k(x_k)-f(x_k)|=1$.\pause
\item La sucesi\'on $p_n(x)=x^n$ \textcolor{red}{no converge uniformemente} en $D=[0,1]$ pues podemos elegir $x_k=2^{-1/k}$, de manera que $f_k(x_k)=1/2$, por lo que $|f_k(x_k)-f(x_k)|=1/2$.\pause
\item Para analizar lo que ocurre con la sucesi\'on $\displaystyle g_n(x)=\frac{x^2+nx}{n}$ con dominio $D=\rz$ veamos primero algunas gr\'aficas
\end{itemize}
\end{frame}


\begin{frame}{Ejemplos}
\begin{figure}
\includegraphics[ height=6.5cm, width=6.5cm]{Potencias2.pdf}
\caption{Gr\'aficas de $\displaystyle g_n(x)=\frac{x^2+nx}{n}$, con $n=1,2, 4, 7$  y $11$ junto con $f(x)=x$}
\end{figure}
\end{frame}


\begin{frame}{Ejemplos}
La explicaci\'on geom\'etrica de la convergencia uniforme sugiere que $g_n$ \textcolor{red}{no converge uniformemente} a $f(x)=x$.
\pause\vskip4pt
Mirando con cuidado a la definici\'on de convergencia podemos dar una condici\'on que implica la \textcolor{blue}{no convergencia uniforme} de una sucesi\'on hacia una funci\'on en un dominio.\pause
\begin{afirma}
La sucesi\'on de funciones $(f_n)$, con $f_n:D\subseteq\rzp\to\rzq$, no converge uniformemente a la funci\'on $f:D\subseteq\rzp\to\rzq$ en $D$ si existe $\epsilon_0>0$, una sucesi\'on $(x_k) \in D$ tal que la subsucesi\'on $(f_{n_k})$ cumple $\|f_{n_k}(x_k)-f(x_k)\|\geq\epsilon_0$
\end{afirma}
\end{frame}


\begin{frame}{Ejemplos}
\begin{figure}
\includegraphics[height=6.5cm, width=6.5cm]{NoConv.pdf}
\caption{Sugerencia para probar la no-convergencia uniforme de $g_n$}
\end{figure}
\end{frame}

\begin{frame}{Convergencia uniforme y norma infinito}
Otro modo de entender la convergencia uniforme que ser\'a \'util cuando trabajemos en el espacio $C(K)$, con $K$ compacto, es recordando la norma 
$$\|f\|_{\infty,D}=\sup\big\{\|f(x)\|:x\in D\big\}$$
para $f$ en el espacio vectorial $B_{p,q}(D)$ que consta de funciones $f:D\subseteq\rzp\to\rzq$ que son acotadas.\pause
\begin{afirma}
Una sucesi\'on en $B_{p,q}(D)$ es uniformemente convergente en $D$ a cierta $f:D\to\rzq$ si y s\'olo si $\|f_n-f\|_{\infty,D}\to0$ si $n\to\infty$.
\end{afirma}\pause
Este nuevo punto de vista sirve para completar la respuesta del ejemplo de la sucesi\'on  $\displaystyle g_n(x)=\frac{x^2+nx}{n}$.
\end{frame}


\begin{frame}{Ejemplo revisitado}
Tomando $D=[0,A]$, con $A>0$ fijo, la sucesi\'on $\displaystyle g_n(x)=\frac{x^2+nx}{n}$ converge en norma a $f(x)=x$, pues
$$\|g_n-f\|_{\infty,D}=\sup\big\{|g_n(x)-f(x)|:0\leq x\leq A\big\}\pause=\sup\left\{\frac{x^2}{n}:0\leq x\leq A\right\}\pause=\frac{A^2}{n}.\pause$$
N\'otese tambi\'en que este razonamiento permite concluir que $g_n$ no converge uniformemente en todo $\rz$, pues el supremo en cada subintervalo $[0,A]$ crece cuando $A$ crece.
\end{frame}


\begin{frame}{Otros ejemplos}
\begin{itemize}
\item Consideremos la sucesi\'on $g_n(x)=x^n(1-x^n)$ para $x\in[0,1]$.\pause
\end{itemize}
\begin{figure}
\includegraphics[height=6.5cm, width=6.5cm]{MonomiosB.pdf}
\caption{Algunas gr\'aficas de $g_n$ con $n=1,2,3,7$ y $11$}
\end{figure}
\end{frame}

\begin{frame}{Otros ejemplos}
De acuerdo a lo visto en las gr\'aficas podemos intuir\pause
\begin{itemize}
\item Que la sucesi\'on converge puntualmente a $g\equiv0$.\pause
\item Que hay un m\'aximo de $g_n$ en $[0,1]$ que est\'a acerc\'andose a 1.\pause
\item Que si $n$ crece, y $x\in[0,1]$ es dado, entonces $g_n(x)$ eventualmente comienza a decaer.\pause
\end{itemize}
Para formalizar o desmentir las intuiciones, observemos que para $x\in[0,1]$ tenemos $|g_n(x)|\leq |x|^n.\quad$\pause Y como $0\leq x\leq1$, entonces $g_n \to g$ puntualmente.
\vskip4pt\pause
Para calcular los m\'aximos, procedemos como en c\'alculo:
$$g_n'(x)=0\;\Leftrightarrow\; x=0\,\text{ \'o }\,x=\frac{1}{2^{1/n}}\pause\qquad g''(2^{-1/n})=-2^{-1/n}n^2<0.\pause$$
De hecho el valor m\'aximo de $g_n$ es $g_n(2^{-1/n})=2^{-1}(1-2^{-1})=1/4$.
\vskip4pt\pause
De aqu\'\i\ que $\|g_n\|_{\infty,[0,1]}=1/4$, por lo que la convergencia de $g_n$ hacia $g\equiv0$ no es uniforme.
\end{frame}

\begin{frame}{Otros ejemplos}
\begin{itemize}
\item Veamos un ejemplo un poco m\'as complicado. Consid\'erese $g\in C[0,1]$ y def\'\i nase la sucesi\'on $(f_n)$ por medio de 
$$f_1(x)=g(x),\qquad f_n(x)=\frac{1}{n}\int_0^xf_{n-1}(t)\,dt\qquad\text{ para }\,x\in[0,1].\pause$$
\end{itemize}
\textcolor{blue}{Se puede probar} (por inducci\'on sobre $n$) que para $x\in[0,1]$ se cumple
$$|f_n(x)|\leq\frac{\|g\|_{\infty,[0,1]}x^{n-1}}{n!}\pause$$
Por tanto $f_n$ converge uniformemente a $f\equiv0$.
\end{frame}

\begin{frame}{Criterio de Cauchy}
Otro uso que podemos dar a la idea de introducir una norma infinito al espacio $B_{p,q}(D)$, $D\subset\rzp$, es el siguiente criterio de Cauchy para la convergencia uniforme. \pause
\begin{teo}
Sea $(f_n)$ una sucesi\'on de funciones en $B_{p,q}(D)$. Entonces existe $f\in B_{p,q}(D)$ tal que $f_n\to f$ uniformemente en $D$ si y s\'olo si dada $\epsilon>0$ existe $M\in\nz$ tal que 
$$m,n\geq M\qquad\text{implica}\quad\|f_n-f_m\|_{\infty,D}<\epsilon$$
\end{teo}\pause

{\color{green!59!black} $\big(\Rightarrow\big)\quad$} \pause Dada $\epsilon>0$ elegimos $K\in\nz$ tal que $n\geq K$ implique $\|f_n-f\|_{\infty,D}<\epsilon/2$
\pause\vskip4pt
Si ahora elegimos $m,n\geq K$ concluiremos
$$\|f_n-f_m\|_{\infty,D}\leq \|f_n-f\|_{\infty,D}+\|f-f_m\|_{\infty,D}<\frac{\epsilon}{2}+\frac{\epsilon}{2}=\epsilon$$
\end{frame}

\begin{frame}{Criterio de Cauchy}

{\color{green!59!black} $\big(\Leftarrow\big)\quad$} \pause Ahora, dada $\epsilon>0$ podemos elegir $M\in\nz$ de manera que $m,n\geq M$ implica $\|f_n-f_m\|_{\infty,D}<\epsilon$.
\pause\vskip4pt
N\'otese que esto implica a su vez que para cada $x\in D$ se tiene $\|f_n(x)-f_m(x)\|<\epsilon$, \pause es decir, que $(f_n(x))$ es una sucesi\'on de Cauchy en $\rzq$.
Podemos entonces definir un l\'\i mite puntual $f(x)=\lim f_n(x)$ para cada $x\in D$. \pause Ahora veremos que la funci\'on as\'\i\ definida es tambi\'en el l\'\i mite uniforme de la sucesi\'on. 
\pause\vskip4pt
Fijando $m\geq M$ en la anterior estimaci\'on y haciendo $n\to\infty$ obtenemos $\|f_n(x)-f_m(x)\|\to\|f(x)-f_m(x)\|$, \pause por lo que $\|f(x)-f_m(x)\|\leq\epsilon$ para $x\in D$ y $m\geq M$. \pause Como de aqu\'\i\ concluimos $\|f(x)\|\leq\epsilon+\|f_M(x)\|$ para $x\in D$, entonces $f\in B_{p,q}(D)$. 
\pause\vskip4pt
Adem\'as lo anterior tambi\'en puede escribirse como $\|f-f_m\|_{\infty,D}<\epsilon$ para $m\geq M$.

\hfill$\blacksquare$
\end{frame}

\begin{frame}{Flashback...}
En el final de la prueba anterior se ha usado el siguiente resultado:

\begin{afirma}
Si una sucesi\'on $(x_n)$ en $\rzp$ converge a cierto $x\in\rzp$ y para ciertos $v\in\rzp$, $r>0$, se cumple $\|x_n-v\|<r$ para $n>J$, entonces $\|x-v\|\leq r$
\end{afirma}\pause

Para demostrar esta afirmaci\'on, definamos el conjunto abierto $V=\big\{y\in\rzp:\|y-v\|>r\big\}$. \pause Entonces $x\not\in V$, pues si $x\in V$ entonces $V$ ser\'\i a una vecindad de $x$, y por convergencia de la sucesi\'on, tendr\'\i amos que $x_N\in V$ para $N>J$, lo cual contradice la hip\'otesis. \pause As\'\i\ que $x\not\in V$, es decir $\|x-v\|\leq r$.

\hfill$\blacksquare$
\end{frame}

\begin{frame}{Nota sobre polinomios de Taylor}\pause
Una versi\'on adaptada del teorema de Taylor cl\'asico establece que 
\begin{afirma}
Sea $f:\rz\to\rz$ tal que todas sus derivadas son continuas y uniformemente acotadas por $M>0$ en un intervalo compacto $[a,b]$. \pause Fijemos $x_0\in[a,b]$, y elijamos $\epsilon>0$ de manera que $[\alpha,\beta]\subset(a,b)$, con $\alpha=x_0-\epsilon/2$, $\beta=x_0+\epsilon/2$. \pause Entonces para cualquier $x\in(\alpha,\beta)$
\begin{align*}
f(x)=f(x_0)&+\frac{f'(x_0)}{1!}(x-x_0)+\frac{f''(x_0)}{2!}(x-x_0)^2+\cdots+\frac{f^{(k)}(x_0)}{k!}(x-x_0)^{k}\\
&+\frac{f^{(k+1)}(\xi)}{(k+1)!}(x-x_0)^{k+1}=:P_{k}(x)+R_k(x)
\end{align*}
\pause para cierta $\xi\in(\alpha,\beta)$.
\end{afirma}
\pause Esto podemos reescribirlo como $|f(x)-P_{k}(x)|=|R_{k}(x)|$. 
\end{frame}

\begin{frame}{Nota sobre polinomios de Taylor}
A partir de la expresi\'on $|f(x)-P_{k}(x)|=|R_{k}(x)|$ y la definici\'on de $R_k(x)$, como la $(k+1)$-\'esima derivada de $f$ es continua en $[\alpha,\beta]\subset[a,b]$ entonces 
$$|f(x)-P_{k}(x)|=|f^{(k+1)}(\xi)||x-x_0|^{k+1}\frac{1}{(k+1)!}\leq M\epsilon^{k+1}\frac{1}{(k+1)!}\pause$$
Esta cantidad converge a cero uniformemente si elegimos $0<\epsilon<1$.
\pause\vskip6pt
Obs\'ervese que en este caso tendr\'\i amos convergencia uniforme de la sucesi\'on de {\em polinomios de Taylor} $(P_k)$ hacia la funci\'on $f$ en $[\alpha,\beta]$. 
\pause\vskip6pt
N\'otese tambi\'en que hay otras hip\'otesis bajo las cuales se puede obtener esta convergencia uniforme.
\end{frame}


\begin{frame}{Otro modo de convergencia}
En otros cursos se considera la {\em convergencia integral cuadr\'atica},  tambi\'en conocida como \textcolor{blue}{convergencia en $L^2$}.
\vskip6pt\pause
Se dice que una sucesi\'on de funciones $f_n:[0,1]\to\rz$ converge en $L^2$ a $f:[0,1]\to\rz$ si 
$$\lim_{n\to\infty}\int_0^1|f_n(t)-f(t)|^2dt=0.\pause$$
Esta convergencia induce una topolog\'\i a en $C[0,1]$ diferente a la que induce la norma infinito, pues en este caso las magnitudes se miden usando la {\em norma $L^2$} definida como $\displaystyle\|f\|_{2,[0,1]}=\left(\int_0^1|f(t)|^2dt\right)^{1/2}$. \pause N\'otese que $\|f\|_{2}\leq\|f\|_{\infty}$ en $[0,1]$.
\vskip6pt\pause
De hecho hay un ejemplo de una sucesi\'on de funciones continuas que es de Cauchy en $L^2$ pero no tiene como l\'\i mite en $L^2$ a una funci\'on continua. 
\end{frame}

\begin{frame}{Otro modo de convergencia}
Def\'\i nase para $x\in[-1,1]$
$$f_n(x)=\left\{\begin{array}{ll}
			0&\text{si }\,-1\leq x\leq -1/n\\
			1+nx&\text{si }\,-1/n\leq x\leq 0\\
			1&\text{si }\,0\leq x\leq 1
		\end{array}\right.$$
\begin{figure}[ht]
    \centering
    % \incfig{Escalera1}
\caption{Esquema de la gr\'afica de $f_n$}
\end{figure}
\end{frame}


\begin{frame}{Otro modo de convergencia}
Esta es una sucesi\'on de funciones continuas que converge puntualmente a una funci\'on que no es continua.
\vskip4pt\pause
Sin embargo, $(f_n)$ es de Cauchy en $L^2$, es decir usando la norma $\displaystyle\|f\|_{2,[-1,1]}$, pues si $m>n$:
\begin{align*}
\|f_n-f_m\|_{2,[-1,1]}&=\left(\int_{-1}^1|f_n(t)-f_m(t)|^2dt\right)^{1/2}\leq\left(\int_{-1/n}^0|f_n(t)|^2dt\right)^{1/2}\\
&\leq\left(\int_{-1/n}^0|1+nt|^2dt\right)^{1/2}=\left(\frac{1}{n}+\frac{2n}{2}\left(\frac{1}{n}\right)^2+\frac{n^2}{3}\left(\frac{1}{n}\right)^3\right)^{1/2}\\
&=\left(\frac{1}{n}+\frac{1}{n}+\frac{1}{3n^2}\right)^{1/2}
\end{align*}
\end{frame}


\begin{frame}{Sucesiones de funciones continuas}
Analizamos ahora el caso en que se tiene una sucesi\'on de funciones continuas.
\vskip6pt\pause
Recordemos que los ejemplos de $p_n(x)=x^n$ en $[0,1]$, y las $f_n$ del ejemplo anterior en $[-1,1]$ muestran que el l\'\i mite puntual de funciones continuas \textcolor{blue}{no necesariamente} es continua.
\vskip6pt\pause
Pero en ambos ejemplos la convergencia no era uniforme. \pause Cabe entonces preguntarse si la convergencia uniforme de funciones continuas da como l\'\i mite a una funci\'on continua \pause

\begin{teo}
Sea $(f_n)$ una sucesi\'on de funciones continuas definidas en $D\subset\rzp$, tomando valores en $\rzq$. Si $f_n\to f$ uniformemente en $D$ entonces $f$ es continua en $D$.
\end{teo}
\end{frame}


\begin{frame}{Prueba del Teorema}
\textcolor{red}{Dada $\epsilon>0$} hallamos $N=N(\epsilon/3)\in\nz$, de manera que $\|f_N(x)-f(x)\|<\epsilon/3$ para $x\in D$. \pause Si por otro lado se toma $a\in D$ arbitrario se tendr\'a
\begin{align*}
\|f(x)-f(a)\|\leq\|f(x)-f_N(x)\|&+\|f_N(x)-f_N(a)\|+\|f(a)-f_N(a)\|\\
&<\frac{\epsilon}{3}+\|f_N(x)-f_N(a)\|+\frac{\epsilon}{3}
\end{align*}\pause
Ahora explotamos el hecho de que $f_N$ es continua en $a$ para \textcolor{red}{determinar $\delta>0$}, que depende de $\epsilon/3$, $a$ y $f_N$, de manera que 
$$\textcolor{red}{\|x-a\|<\delta}\qquad\text{implica}\qquad\|f_N(x)-f_N(a)\|<\frac{\epsilon}{3}\pause$$
En conclusi\'on, para tales $x$ tendremos \textcolor{red}{$\|f(x)-f(a)\|<\epsilon$}.\hfill$\blacksquare$
\end{frame}

\begin{frame}{Sucesiones de funciones continuas}
El teorema anterior puede reformularse en el lenguaje de normas uniformes.\pause
\begin{teo}
Si $(f_n)$ es una sucesi\'on en $BC_{p,q}(D)$ tal que $\|f_n-f\|_{\infty,D}\to0$ entonces $f\in BC_{p,q}(D)$.\pause
\end{teo}

Por otro lado el rec\'\i proco de este teorema es falso: la siguiente sucesi\'on $(\varphi_n)$ consta de funciones continuas y acotadas en $(0,\infty)$, pero converge a una funci\'on que no es acotada en $(0,\infty)$; \pause debe entonces ocurrir que la convergencia no puede ser uniforme:
$$\varphi(x)=\left\{\begin{array}{ll}
				1/x&\text{si }\,x\in[1/n,\infty)\\
				n&\text{si }\,x\in(0,1/n)
			\end{array}\right.$$
\end{frame}

\begin{frame}{Una especie de rec\'\i proco}
Curiosamente, al a\~nadir alguna hip\'otesis extra podemos concluir un rec\'\i proco del teorema anterior.\pause

\begin{teo}[Dini]
Sea $(f_n)$ una sucesi\'on de funciones continuas definidas en un espacio m\'etrico compacto $K$. \pause Sup\'ongase adem\'as que para cualquier $x\in K$ ocurre que $f_n(x)$ converge a $f(x)$ \textcolor{blue}{puntualmente} \underline{como una sucesi\'on decreciente}, para cierta $f\in C(K)$. Entonces $f_n\to f$ uniformemente\pause
\end{teo}
Para demostrar el teorema supongamos SPG que la sucesi\'on es puntualmente decreciente hacia la funci\'on $f\equiv0$ y que $f_n(x)\geq0$ para toda $x\in K$.
\pause\vskip6pt
Esto es cierto, pues si $f_n$ converge puntualmente de manera decreciente a $f(x)\not\equiv0$, entonces ocurrir\'a que $f_n(x)\geq f(x)$ y entonces podr\'\i amos trabajar con $F(x):=f_n(x)-f(x)\geq 0$ para todo $x\in K$, \pause que cumple $F_n$ converge a la funci\'on constante $0$ puntualmente de manera decreciente.\hfill$\square$
\end{frame}


\begin{frame}{Una especie de rec\'\i proco}
Bajo estas suposiciones, si no se tuviera convergencia uniforme de $f_n$ hacia $f\equiv0$, existir\'\i a $\epsilon_0>0$ y $(x_n)$ sucesi\'on en $K$ tales que \textcolor{red}{$f_n(x_n)\geq\epsilon_0$}
\pause\vskip6pt
Como la sucesi\'on $(x_n)$ est\'a en el cerrado y acotado $K$, podemos tomar una subsucesi\'on $(\tilde x_n)$ convergente, digamos a $x^*\in K$
\pause\vskip6pt
Por otro lado, la hip\'otesis de convergencia implica que existe $M\in\nz$ tal que $f_m(x^*)<\epsilon_0$ para $m\geq M$.
\pause\vskip6pt
Y por la continuidad de $f_M$ existe otro $N\in\nz$ tal que $n\geq N$ implica $f_M(\tilde x_n)<\epsilon$.
\pause\vskip6pt
Tomando $N>M$ obtendremos $\textcolor{red}{f_N(\tilde x_N)}\leq f_M(\tilde x_N)\textcolor{red}{<\epsilon_0}$ \pause lo cual contradice la construcci\'on de la subsucesi\'on.\hfill$\blacksquare$ 
\end{frame}




%%%%%%%%%%%%%%%%%%%%%%%%%%%%%%%%%%%%%%%%%%%%%%%%%%%%%%%%%%%
%%%%%%%%%%%%%%%%%%%%%%%%%%%%%%%%%%%%%%%%%%%%%%%%%%%%%%%%%%%
%%%%%%%%%%%%%%%%%%%%%\end{document}%%%%%%%%%%%%%%%%%%%%%%%%%%%%%%
%%%%%%%%%%%%%%%%%%%%%%%%%%%%%%%%%%%%%%%%%%%%%%%%%%%%%%%%%%%
%%%%%%%%%%%%%%%%%%%%%%%%%%%%%%%%%%%%%%%%%%%%%%%%%%%%%%%%%%%






\begin{frame}{Idea de la Teor\'\i a de Aproximaci\'on}

Podemos ahora plantear una idea b\'asica de la teor\'\i a de aproximaci\'on, tratando de ver desde otra perspectiva los resultados anteriores.
\vskip6pt\pause
Dadas $f,g:D\subseteq\rzp\to\rzq$, se dice que \textcolor{blue}{$g$ aproxima uniformemente a $f$ con error $\epsilon>0$} si $\|f-g\|_{\infty,D}<\epsilon$.
\vskip6pt\pause
Dada $\mathcal F$ una familia de funciones de $D\subseteq\rzp$ en $\rzq$ y $f:D\to\rzq$, se dice que \textcolor{blue}{$f$ es aproximada uniformemente en $D$ por elementos de $\mathcal F$} si para toda $\epsilon>0$ existe $g_\epsilon\in\mathcal F$ tal que $\|f-g_\epsilon\|_{\infty,D}<\epsilon$.
\vskip8pt\pause
 De este modo, una idea que perseguiremos por un rato es iniciar con $f\in C(X)$, con $X$ espacio m\'etrico, y buscar/caracterizar familias $\mathcal F\subset C(X)$ cuyos elementos aproximen uniformemente a $f$ en $X$.
\end{frame}


\begin{frame}{Aproximaci\'on por funciones escalera}
Una funci\'on $g:\rzp\to\rzq$ es una \textcolor{red}{funci\'on escalera} sobre $D\subset\rzp$ si toma un n\'umero finito de valores diferentes, y los valores distintos de cero los toma en celdas acotadas de $\rzp$.
\vskip6pt\pause
Recordemos que 
\begin{itemize}
\item Una \textcolor{red}{celda acotada} en $\rzp$ es un conjunto de la forma $I=I_1\times I_2\times\cdots\times I_p$ donde cada $I_j$ es un intervalo acotado de $\rz$.\pause
\end{itemize} 
Denotemos por $\Sigma(D)$ a la familia de funciones escalera sobre $D\subseteq\rzp$. \pause
\begin{teo}
Sea $f:J\subset\rzp\to\rzq$ una funci\'on continua en $J$, que suponemos una celda cerrada y acotada. Entonces $f$ puede aproximarse uniformemente en $J$ por elementos de $\Sigma(J)$. 
\end{teo}
\end{frame}


\begin{frame}{Aproximaci\'on por funciones escalera}
Al ser $f$ uniformemente continua en $J$, dada $\epsilon>0$ elegimos $\delta>0$ tal que $x,y\in J$ con $\|x-y\|<\delta$ implica $\|f(x)-f(y)\|<\epsilon$.
\pause\vskip4pt
Dividimos $J$ en celdas disjuntas y acotadas $I_1,I_2,\dots,I_N$, tales que $x,y\in I_k$ implique $\|x-y\|<\delta$
\begin{figure}[ht]
    \centering
    % \incfig{BolaComeCelda}
\caption{Bisectando los intervalos que conforman $J$ construimos las celdas $I_k$ de manera que dos puntos en ella distan entre s\'\i\ menos que $\delta$}
\end{figure}
\end{frame}


\begin{frame}{Aproximaci\'on por funciones escalera}
Ahora elegimos y fijamos $x_k\in I_k$ ($k=1,2,\dots,N$) y definimos la siguiente funci\'on escalera
$$g_\epsilon(x)=\left\{\begin{array}{ll}
					f(x_k)&\text{si }\,x\in I_k\\
					0&\text{si }\,x\not\in J
				\end{array}\right.\pause$$
\begin{figure}[ht]
    \centering
    % \incfig{Tubular-Escalera}
\caption{Esquema de la gr\'afica de la funci\'on escalera $g_\epsilon$}
\end{figure}
\end{frame}


\begin{frame}{Aproximaci\'on por funciones escalera}
Entonces $\|f-g_\epsilon\|_{\infty,J}<\epsilon$, como veremos a continuaci\'on.
\pause\vskip5pt
Dada $x\in J$ detectamos $I_k$ tal que $x\in I_k$, de manera que $g_\epsilon(x)=f(x_k)$ para la $x_k\in I_k$ elegida en la construcci\'on de $g_\epsilon$.
\pause\vskip5pt
Entonces como $x,x_k\in I_k$, de nuevo por construcci\'on 
$$|x-x_k|<\delta\qquad\text{lo cual implica }\quad |f(x)-g_\epsilon(x)|=|f(x)-f(x_k)|<\epsilon$$
\hfill$\blacksquare$
\end{frame}


\begin{frame}{Aproximaci\'on por funciones lineales por pedazos}
Decimos que $g:J=[a,b]\to\rz$ es \textcolor{red}{lineal por pedazos} si existen $(n+1)$ puntos $c_k\in J$ cumpliendo $a=c_0<c_1<\cdots<c_n=b$, \pause y para cada $k$ dos n\'umeros $A_k,B_k\in\rz$ tales que para $x\in[c_{k-1},c_k]$ se tiene $g(x)=A_kx+B_k$.
\vskip8pt\pause
Obs\'ervese que para que una funci\'on lineal por pedazos sea continua, los coeficientes $A_k$, $B_k$ deben cumplir ciertas condiciones. 
\vskip4pt\pause
Por ejemplo, supongamos que se tienen s\'olo los intervalos $[a,c_1]$ y $[c_1,b]$.  
\pause Entonces en $x=c_1$ se debe cumplir que 
$$A_1c_1+B_1=A_2c_1+B_2\pause$$
cumpli\'endose condiciones similares al haber m\'as puntos en la {\em partici\'on} que inducen los $c_k$.
\end{frame}


\begin{frame}{Aproximaci\'on por funciones lineales por pedazos}
\begin{teo}
Si $f:J=[a,b]\to\rz$ es continua en el intervalo compacto $J$ entonces puede aproximarse uniformemente en $J$ por funciones lineales por pedazos continuas. \pause
\end{teo}
\begin{figure}[ht]
    \centering
    % \incfig{Tubular-Lineal}
\caption{Esquema de la gr\'afica de la funci\'on lineal por pedazos}
\end{figure}
\end{frame}


\begin{frame}{Aproximaci\'on por funciones lineales por pedazos}
La demostraci\'on del teorema es de nuevo constructiva. 
\pause\vskip5pt 
Dada $\epsilon>0$ determinamos $\delta>0$ (por la continuidad uniforme de $f$) de manera que $|x-y|<\delta$ implique $|f(x)-f(y)|<\epsilon$.
\pause\vskip5pt 
Ahora dividimos $J$ usando una partici\'on  $\{a=c_0<c_1<c_2<\cdots<b=c_N\}$ tal que $c_k-c_{k-1}<\delta$ para toda $k$ 
\pause\vskip5pt 
La idea final es unir los puntos $(c_k,f(c_k))$ por medio de lineas rectas para definir la funci\'on $g_\epsilon$ que aproxime uniformemente a $f$ en $J$.
\pause\vskip5pt 
Aunque \'esto se ve clar en la gr\'afica daremos detalles de esta demostraci\'on
\end{frame}


\begin{frame}{Aproximaci\'on por funciones lineales por pedazos}
Si $x\in J$ y $x\neq c_k$ para todo $k$ (pues de otro modo no hay nada que probar),  supongamos que $c_{j-1}<x<c_j$. \pause Entonces 
$$|f(x)-g_\epsilon(x)|\leq|f(x)-f(c_j)|+|f(c_j)-g_\epsilon(x)|\pause<\epsilon+|f(c_j)-g_\epsilon(x)|\pause$$
Ahora recordemos que $f(c_j)$ y $g_\epsilon(x)$ son dos valores de una recta de la forma $Ax+B$ en $[c_{j-1},c_j]$.
\pause\vskip5pt
Entonces 
\begin{align*}
|f(c_j)-g_\epsilon(x)|=\textcolor{red}{|Ac_j+B-Ax-B|}&\textcolor{red}{\leq|Ac_j+B-Ac_{j-1}-B|}\\
&=|f(c_j)-f(c_{j-1})|<\epsilon
\end{align*}
\hfill$\blacksquare$
\end{frame}


\begin{frame}{Flashback...}
\begin{afirma}
En general, si en un intervalo $[a,b]$ se define la funci\'on lineal $h(x)=\alpha x+\beta$, y se toma $x\in (a,b)$ entonces
$$|h(b)-h(x)|\leq|h(b)-h(a)|$$
\end{afirma}
\pause

Sabemos por el Teorema del Valor Medio de c\'alculo que 
$$|h(b)-h(x)|=|h'(\xi)||b-x|\pause=\frac{|h(b)-h(a)|}{b-a}|b-x|\pause\leq|h(b)-h(a)|$$
\end{frame}

\begin{frame}{Observaciones}\pause
Notemos que los dos teoremas anteriores establecen que cualquier funci\'on $f:[a,b]\to\rz$ continua en el intervalo compacto $[a,b]$ puede aproximarse uniformemente por funciones definidas por pedazos\pause
\begin{itemize}
\item En el primer caso en cada porci\'on del intervalo la funci\'on aproximante es un polinomio de grado 0\pause
\item En el segundo la funci\'on aproximante es un polinomio de grado 1 en cada porci\'on del intervalo.
\end{itemize}
\pause\vskip6pt 
Por otro lado, en la prueba del teorema anterior usamos una simple observaci\'on geom\'etrica: Dados dos puntos, hay una (\'unica) recta (polinomio de grado 1) que los une. \pause Tambi\'en, si tenemos tres puntos hay una (\'unica) par\'abola que los une.
\pause\vskip5pt 
Cabe preguntar: >Habr\'a un polinomio de orden $n$ que une a $n+1$ puntos dados en el plano?
\pause\vskip5pt 
En el mismo tenor, se podr\'\i a uno preguntar si para una funci\'on $f:[a,b]\to\rz$ continua en el intervalo compacto $[a,b]$ existe un polinomio que la aproxima uniformemente en $J$.
\end{frame}


\begin{frame}{Problema de Interpolaci\'on -- Soluci\'on de Lagrange}
\begin{afirma}
Nos son dados $(n+1)$ puntos en el plano $\rz^2$
$$(x_0,y_0),\,(x_1,y_1),\dots,\,(x_n,y_n)\qquad\text{cumpliendo }\,x_0<x_1<\cdots<x_n.\pause$$
Hallar un polinomio $p(x)$ de grado menor o igual a $n$ tal que $p(x_k)=y_k$ para $k=0,1,2,\dots,n$. \pause
\end{afirma}
Con un poco de astucia uno puede comenzar construyendo un polinomio $L_j$, $j=0,1,2,\dots,n$, tal que $L_j(x_i)=0$ si $i\neq j$ y $L_j(x_i)=1$ si $i=j$.
\pause\vskip5pt
En efecto sea
$$L_j(x)=\frac{(x-x_0)(x-x_1)\cdots(x-x_{j-1})(x-x_{j+1})\cdots(x-x_n)}{(x_j-x_0)(x_j-x_1)\cdots(x_j-x_{j-1})(x_j-x_{j+1})\cdots(x_j-x_n)}\pause$$
Luego basta definir
$$L(x)=\sum_{k=0}^ny_kL_k(x)$$
\end{frame}


\begin{frame}{Problema de Interpolaci\'on}
As\'\i\ que la afirmaci\'on que resultar\'\i a algo m\'as dif\'\i cil de establecer es la \textcolor{red}{unicidad}. \pause\vskip5pt Hay una prueba corta usando el llamado \textcolor{blue}{Teorema Fundamental del \'Algebra}:
\begin{afirma}
Todo polinomio  en una variable, distinto de la constante cero, de grado $n$ con coeficientes complejos tiene $n$ ra\'\i ces, contando sus multiplicidades.\pause
\end{afirma}
As\'\i\ que cuando un polinomio de grado a lo m\'as $n$ tenga $n+1$ ra\'\i ces distintas, entonces tendr\'a que ser el polinomio constante cero.
\vskip5pt\pause
Sup\'ongase que hay dos polinomios $P$, $Q$ de grado menor o igual a $n$ tales que $P(x_k)=Q(x_k)=y_k$ para $k=0,1,2,\dots,n$.
\vskip5pt\pause
Entonces $D(x)=P(x)-Q(x)$ es un polinomio de grado a lo m\'as $n$ que cumple $D(x_k)=P(x_k)-Q(x_k)=0$ para $k=0,1,2,\dots,n$, \pause o sea con $n+1$ ra\'\i ces. Entonces $D\equiv0$\hfill$\blacksquare$
\end{frame}


\begin{frame}{Aproximaci\'on por polinomios}
Ahora damos una primera prueba del teorema de Weierstrass en el intervalo $[0,1]$
\begin{teo}[Weierstrass]
Si $f:[0,1]\to\rz$ es continua en el intervalo compacto $[0,1]$ entonces existe una sucesi\'on de polinomios $(p_n)$ tales que $p_n\to f$ uniformemente en $[0,1]$.\pause
\end{teo}
Suponemos SPG que $f(0)=f(1)=0$, pues una vez que lo demostremos en este caso podemos ya asumir que $f(0)$ y $f(1)$ son m\'as arbitrarios y considerar
$$g(x)=f(x)-f(0)-x[f(1)-f(0)],\qquad0\leq x\leq1\pause$$
Esta funci\'on cumple $g(0)=g(1)=0$. \pause Adem\'as $p(x)=f(x)-g(x)$ es un polinomio, y si $(p_n)$ aproxima uniformemente a $g$ entonces $p+p_n$ aproxima a $f$ uniformemente en $[0,1]$ pues 
$$\|f-(p+p_n)\|_{\infty}=\|f-(f-g)-p_n\|_{\infty}=\|g-p_n\|_\infty\to0$$
\end{frame}


\begin{frame}{Aproximaci\'on por polinomios}
Ahora, adem\'as de suponer  $f(0)=f(1)=0$, extenderemos $f$ como cero fuera de $[0,1]$, de manera que es uniformemente continua en $\rz$.
\pause\vskip5pt
Definamos para $x\in\rz$
$$Q_n(x)=c_n\big(1-x^2\big)^n,\qquad n=1,2,3,\dots\pause$$
donde $c_n>0$ se elige de manera que $\displaystyle\int_{-1}^1Q_n(t)\,dt=1$, \pause y en particular  $\displaystyle f(x)=\int_{-1}^1Q_n(t)f(x)\,dt$.
\vskip-1cm
\begin{figure}[ht]
\includegraphics[ height=3.5cm, width=4.5cm]{PolinomiosQn.pdf}
\caption{Gr\'aficas \underline{ilustrativas y aproximadas} de algunos $Q_n(x)$}
\end{figure}
\end{frame}


\begin{frame}{Aproximaci\'on por polinomios}
Definimos para $0\leq x\leq 1$
$$p_n(x)=\int_{-1}^{1}f(x+t)Q_n(t)\,dt\qquad n=1,2,3,\dots\pause$$
Por medio de cambios de variable demostraremos que $p_n(x)$ es un polinomio.
\vskip5pt\pause
Como $0\leq x+t\leq 1$ entonces $-x\leq t\leq 1-x$, por lo que 
$$\int_{-1}^{1}f(x+t)Q_n(t)\,dt=\int_{-x}^{1-x}f(x+t)Q_n(t)\,dt\pause=\int_{0}^{1}f(s)Q_n(s-x)\,ds\pause$$
donde en la \'ultima integral se hizo el cambio de variable $s=x+t$. \pause
Entonces 
$$p_n(x)=c_n\int_{0}^{1}f(s)\big[1-(s-x)^2\big]^n\,ds\pause=c_n\int_{0}^{1}f(s)\,ds-c_n\int_{0}^{1}P\big((s-x)^2\big)\,ds\pause$$
donde $P(u)$ es una expresi\'on polinomial. \pause En conclusi\'on $p_n(x)$ es un polinomio. 
\end{frame}


\begin{frame}{Aproximaci\'on por polinomios}
Para establecer la convergencia uniforme de $p_n$ hacia $f$ hacemos la siguiente estimaci\'on
$$|p_n(x)-f(x)|=\left|\int_{-1}^{1}\big[f(x+t)-f(x)\big]Q_n(t)\,dt\right|\pause\leq\int_{-1}^{1}\big|f(x+t)-f(x)\big|\,Q_n(t)\,dt\pause$$
donde hemos usado las propiedades de $Q_n$.
\vskip5pt\pause
Ahora hacemos una separaci\'on de la integral para usar la continuidad de $f$; \pause supongamos que $|f(x)|\leq M$ para todo $x\in [0,1]$, \pause y que para $\epsilon>0$ elegimos la $\delta>0$ adecuada de la continuidad uniforme de $f$. \pause Entonces
\begin{align*}
\int_{-1}^{1}&\big|f(x+t)-f(x)\big|\,Q_n(t)\,dt=\left[\int_{-\delta}^{\delta}+\int_{\delta\leq|t|\leq1}\right]\big|f(x+t)-f(x)\big|\,Q_n(t)\,dt\\
&\qquad<\epsilon+2M\int_{\delta\leq|t|<1}\,Q_n(t)\,dt=\epsilon+2M\left[\int_{-1}^{-\delta}+\int_{\delta}^{1}\right]Q_n(t)\,dt
\end{align*}
\end{frame}


\begin{frame}{Aproximaci\'on por polinomios}
Ahora debemos estimar $Q_n(t)$ para $\delta\leq|t|\leq1$.
\vskip5pt\pause
Para esto acotaremos por arriba a $c_n$. \pause N\'otese que 
$$\int_{-1}^{1}\big(1-x^2\big)^ndx=2\int_{0}^{1}\big(1-x^2\big)^ndx\pause\geq2\int_{0}^{1/\sqrt n}\big(1-x^2\big)^ndx\pause$$
Tambi\'en pude verse que \textcolor{blue}{$\big(1-x^2\big)^n\geq 1-nx^2$} \pause por lo que continuamos 
$$\geq2\int_{0}^{1/\sqrt n}\big(1-nx^2\big)\,dx\pause=\frac{4}{3\sqrt n}>\frac{1}{\sqrt n}\pause$$
Esto implica que $c_n<\sqrt n$ al multiplicar por $c_n$:
$$1=c_n\int_{-1}^{1}\big(1-x^2\big)^ndx>\frac{c_n}{\sqrt n}$$
\end{frame}


\begin{frame}{Aproximaci\'on por polinomios}
Lo anterior nos lleva a que $c_n\leq\sqrt n$.
\pause\vskip5pt
De aqu\'\i\ que para $\delta\leq|t|\leq1$ tendremos 
$$Q_n(t)\leq\sqrt n(1-t^2)^n=\sqrt n(1-|t|^2)^n\leq\sqrt n(1-\delta^2)^n$$
\pause
Esto tiene dos consecuencias: primero, $Q_n\to0$ uniformemente en $\delta\leq|t|\leq1$, pues de hecho \textcolor{blue}{$\sqrt n(1-\delta^2)^n\to0$}
\vskip5pt\pause
Segundo, de vuelta a la integral que est\'abamos estimando 
$$2M\int_{\delta\leq|t|<1}\,Q_n(t)\,dt\leq 4M\sqrt n(1-\delta^2)^n\to 0\pause$$
En conclusi\'on, para cualquier $x\in \intunit$ y $n$ suficientemente grande
$$|p_n(x)-f(x)|\leq\left[\int_{-\delta}^{\delta}+\int_{\delta\leq|t|\leq1}\right]\big|f(x+t)-f(x)\big|\,Q_n(t)\,dt<2\epsilon$$
\hfill$\blacksquare$
\end{frame}


\begin{frame}{Flashback 1...}
\begin{afirma}
$\big(1-x^2\big)^n\geq 1-nx^2$ para $x\in(0,1)$\pause
\end{afirma}
Esta afirmaci\'on se prueba notando que $\big(1-x^2\big)^n- 1+nx^2$ vale $0$ en $x=0$ y tiene derivada positiva en $(0,1)$.\pause
\begin{figure}[ht]
\includegraphics[ height=4.5cm, width=4.5cm]{BernoulliLike.pdf}
\caption{Gr\'aficas de $\big(1-x^2\big)^n- 1+nx^2$, con $n=17$ y $n=22$}
\end{figure}
\end{frame}

\begin{frame}{Flashback 2...}
La siguiente afirmaci\'on est\'a tomada de [Bartle-Sherbert, Theorem 3.2.11]  
\begin{afirma}
Si una sucesi\'on $(x_n)$ de reales positivos cumple que $\displaystyle t_n=\frac{x_{n+1}}{x_n}\to A$ y $0\leq A<1$ entonces $x_n\to0$\pause
\end{afirma}
En nuestro caso 
$$\frac{\sqrt{n+1}(1-\delta^2)^{n+1}}{\sqrt n(1-\delta^2)^n}=(1-\delta^2)\frac{\sqrt{n+1}}{\sqrt n}\pause\to(1-\delta^2)<1\pause$$
y por esto es que $\sqrt n(1-\delta^2)^n\to0$.
\end{frame}

\begin{frame}{Un cambio de variable}
A partir de ahora denotaremos por $\mathbb I$ al intervalo $[0,1]$.
\vskip5pt\pause
Observemos que el haber obtenido el resultado para funciones $f$ definidas en $\intunit$ nos lleva de inmediato a la soluci\'on en el intervalo compacto $[a,b]$, con $a<b$.
\vskip5pt\pause
Para \'esto, dada $g:[a,b]\to\rz$, basta definir una nueva funci\'on 
$$f(t)=g\big((b-a)t+a\big)\quad\text{ para }\quad t\in\intunit.\pause$$
Si el teorema est\'a demostrado para funciones como $f$ entonces \textcolor{blue}{$\|f-p_n\|_{\infty,\intunit}\to0$}. \pause Proponemos entonces que $\widetilde p_n(x)=p_n\big(\frac{x-a}{b-a}\big)$, con $x\in[a,b]$, converge uniformemente a $g$ en $[a,b]$.
\vskip5pt\pause
Para verificar \'esto, basta notar que $g(x)=f\big(\frac{x-a}{b-a}\big)$ para $x\in[a,b]$ \pause y entonces
$$\|g-\widetilde p_n\|_{\infty,[a,b]}=\sup_{x\in[a,b]}|g(x)-\widetilde p_n(x)|\pause=\sup_{t\in[0,1]}|f(t)-p_n(t)|\pause=\textcolor{blue}{\|f-p_n\|_{\infty,\intunit}}$$
\end{frame}


\begin{frame}{Un corolario del Teorema de Weierstrass}
\begin{cor}
Para todo intervalo de la forma $[-a,a]$ (con $a>0$) existe una sucesi\'on de polinomios $(p_n)$ tales que $p_n(0)=0$ para toda $n$, y tal que
$$\lim_{n\to\infty}p_n(x)=|x|\qquad\text{uniformemente en }\,[-a,a].$$
\end{cor}
\pause El teorema de Weierstrass y el cambio de variable antes explicado nos obsequian una sucesi\'on $(p_n^*)$ que converge uniformemente a $v(x)=|x|$ en $[-a,a]$.
\vskip5pt\pause
Como en particular tenemos que $p_n^*(0)\to0$ entonces podemos definir $p_n(x)=p_n^*(x)-p_n^*(0)$, y el corolario queda probado.\hfill$\blacksquare$
\end{frame}


\begin{frame}{Polinomios de Bernstein:}{Otro esquema de aproximaci\'on por polinomios}
Revisamos una {\em ``versi\'on discreta''} del teorema anterior, donde se construyen expl\'\i citamente polinomios de aproximaci\'on.
\vskip5pt\pause
Iniciamos de nuevo con $f:\intunit\to\rz$ y definimos el \textcolor{red}{$n$-\'esimo polinomio de Bernstein} asociado a $f$ como
$$B_n(x)\equiv B_nf(x)=\sum_{k=0}^nf\big(\frac{k}{n}\big){n\choose k}x^k(1-x)^{n-k},\qquad x\in\intunit.\pause$$
\begin{teo}[Bernstein]
Sea $f:\intunit\to\rz$ continua en $\intunit$. Entonces la sucesi\'on $(B_nf)$ de polinomios de Bernstein asociados a $f$ converge uniformemente a $f$ en $\intunit$.
\end{teo}
\end{frame}


\begin{frame}{Polinomios de Bernstein}
Antes de demostrar el teorema, podemos calcular los primeros polinomios de Bernstein. 
\vskip5pt\pause
Recordemos que el teorema del binomio establece que
$$(s+t)^n=\sum_{k=0}^n{n\choose k}s^kt^{n-k},\qquad\text{donde }\quad{n\choose k}=\frac{n!}{k!(n-k)!}.\pause$$
As\'\i\ que al sustituir $s=x$, $t=1-x$ obtenemos
\begin{afirma}
$$1=\sum_{k=0}^n{n\choose k}x^k(1-x)^{n-k}=B_n\varphi_0(x).$$ 
\end{afirma}\pause
De aqu\'\i\ en adelante denotamos por $\varphi_n(x)=x^n$ para $n=0,1,2\dots$
\end{frame}


\begin{frame}{Polinomios de Bernstein}
El polinomio de Bernstein $B_n\varphi_1(x)$ se deduce tomando de nuevo la expresi\'on $\displaystyle (s+t)^n=\sum_{k=0}^n{n\choose k}s^kt^{n-k}$, \pause
y derivando respecto a $s$ obtenemos
$$n(s+t)^{n-1}=\sum_{k=0}^n{n\choose k}ks^{k-1}t^{n-k}.\pause$$
Sustituyendo $s=x$, $t=1-x$ obtenemos
$$n=\sum_{k=0}^n k{n\choose k}x^{k-1}(1-x)^{n-k},\pause$$
y al multiplicar por $x$ nos lleva a 
\begin{afirma}
$$x=\sum_{k=0}^n\big(\frac{k}{n}\big){n\choose k}x^{k}(1-x)^{n-k}\pause=B_n\varphi_1(x).$$
\end{afirma}
\end{frame}


\begin{frame}{Polinomios de Bernstein}
Denotando por $\varphi_2(x)=x^2$, ahora hallaremos $B_n\varphi_2(x)$.
\vskip5pt\pause
Volvemos a la expresi\'on $\displaystyle n(s+t)^{n-1}=\sum_{k=0}^n{n\choose k}ks^{k-1}t^{n-k}.\pause$ y la derivamos de nuevo respecto a $s$:
$$n(n-1)(s+t)^{n-2}=\sum_{k=0}^n{n\choose k}k(k-1)s^{k-2}t^{n-k},\pause$$
sustituimos $s=x$, $t=1-x$
$$n(n-1)=\sum_{k=0}^nk(k-1){n\choose k}x^{k-2}(1-x)^{n-k}.$$
\end{frame}


\begin{frame}{Polinomios de Bernstein}
Multiplicamos $\displaystyle n(n-1)=\sum_{k=0}^nk(k-1){n\choose k}x^{k-2}(1-x)^{n-k}$ por $x^2$ y hacemos manipulaciones algebr\'aicas para llegar a \pause
$$n^2\left(1-\frac{1}{n}\right)x^2=\sum_{k=0}^nk^2{n\choose k}x^{k}(1-x)^{n-k}-\sum_{k=0}^nk{n\choose k}x^{k}(1-x)^{n-k}.\pause$$
Llegamos entonces a
$$\left(1-\frac{1}{n}\right)x^2=\sum_{k=0}^n\big(\frac{k^2}{n^2}\big){n\choose k}x^{k}(1-x)^{n-k}-\frac{1}{n}\sum_{k=0}^n\big(\frac{k}{n}\big){n\choose k}x^{k}(1-x)^{n-k},\pause$$
es decir $\displaystyle \left(1-\frac{1}{n}\right)x^2=B_n\varphi_2(x)-\frac{1}{n}B_n\varphi_1(x)\pause=B_n\varphi_2(x)-\frac{1}{n}x$. \pause Despejando:
\begin{afirma}
$$B_n\varphi_2(x)=\left(1-\frac{1}{n}\right)x^2+\frac{1}{n}x.$$
\end{afirma}
\end{frame}


\begin{frame}{Polinomios de Bernstein}
Para finalmente proceder con la prueba del teorema haremos una \'ultima observaci\'on.
\vskip5pt\pause
Multiplicamos $\displaystyle1=\sum_{k=0}^n{n\choose k}x^k(1-x)^{n-k}\qquad$ por $x^2$, \pause\vskip4pt  
multiplicamos $\displaystyle x=\sum_{k=0}^n\big(\frac{k}{n}\big){n\choose k}x^{k}(1-x)^{n-k}\qquad$ por $-2x$, \pause\vskip4pt
y los sumamos:
$$\textcolor{red}{-x^2= \sum_{k=0}^n\big(x^2-2x\frac{k}{n}\big){n\choose k}x^{k}(1-x)^{n-k}.}\pause$$
Recordemos que 
$$\textcolor{green}{\sum_{k=0}^n\big(\frac{k^2}{n^2}\big){n\choose k}x^{k}(1-x)^{n-k}=\left(1-\frac{1}{n}\right)x^2+\frac{1}{n}x\pause=x^2-\frac{1}{n}(x^2-x).}$$
\end{frame}


\begin{frame}{Polinomios de Bernstein}
Sumando la expresi\'on en \textcolor{red}{rojo} con la que est\'a en \textcolor{green}{verde} obtenemos
$$-\frac{1}{n}(x^2-x)=\sum_{k=0}^n\big(x-\frac{k}{n}\big)^2{n\choose k}x^{k}(1-x)^{n-k},\pause$$
que reescribimos
\begin{afirma}
$$\frac{x(1-x)}{n}=\sum_{k=0}^n\big(x-\frac{k}{n}\big)^2{n\choose k}x^{k}(1-x)^{n-k},$$
\end{afirma}
\end{frame}


\begin{frame}{Demostraci\'on del Teorema de Bernstein}
Observemos primero que la f\'ormula para $B_n\varphi_0$ implica que 
$$f(x)=\sum_{k=0}^nf(x){n\choose k}x^k(1-x)^{n-k}\pause$$
por lo que para $x\in\intunit$
$$\big|f(x)-B_n(x)\big|\leq\sum_{k=0}^n\left|f(x)-f\big(\frac{k}{n}\big)\right|{n\choose k}x^k(1-x)^{n-k}.\pause$$
Ahora suponemos que $\|f\|_{\infty,\intunit}\leq M$ y separaremos los \'\i ndices $k$ de la suma de acuerdo a la cercan\'\i a de $\displaystyle\frac{k}{n}$ a $x$.
\end{frame}


\begin{frame}{Demostraci\'on del Teorema de Bernstein}
Dada $\epsilon>0$ elegimos una $\delta(\epsilon)>0$ que funcione en la definici\'on de continuidad uniforme de $f$ en $\intunit$.
\pause\vskip5pt
Tambi\'en elegimos $n\in\nz$ tal que $\displaystyle n\geq\max\left\{\frac{1}{\delta^4} \,\,,\,\frac{M^2}{\epsilon^2}\right\}$
\pause\vskip5pt
Para dividir la suma en dos sumas, usamos la siguiente notaci\'on:\pause

En $\displaystyle\sum_1$ agrupamos las $k$ tales que $\displaystyle\left|x-\frac{k}{n}\right|<\frac{1}{n^{1/4}}\leq\delta$,\pause

y en $\displaystyle\sum_2$ agrupamos las $k$ restantes.
\pause\vskip5pt
Entonces 
$$\sum_1\left|f(x)-f\big(\frac{k}{n}\big)\right|{n\choose k}x^k(1-x)^{n-k}<\epsilon B_n\varphi_0(x)\pause=\epsilon$$
\end{frame}


\begin{frame}{Demostraci\'on del Teorema de Bernstein}
Ahora usamos el hecho de que $\displaystyle x(1-x)\leq\frac{1}{4}$,  que por la elecci\'on de $n$ tenemos $n\geq\frac{M^2}{\epsilon^2}$, y que en esta sumatoria tenemos $\displaystyle\left|x-\frac{k}{n}\right|\geq\frac{1}{n^{1/4}}$ y obtenemos\pause
\begin{align*}
\uncover<2->{\sum_2\left|f(x)-f\big(\frac{k}{n}\big)\right|&{n\choose k}x^k(1-x)^{n-k}\leq 2M\sum_2{n\choose k}x^k(1-x)^{n-k}\\}
\uncover<3->{&\leq2M\sum_2\frac{(x-k/n)^2}{(x-k/n)^2}{n\choose k}x^k(1-x)^{n-k}\\}
\uncover<4->{&\leq2M\sqrt n\,\textcolor{blue}{\sum_{k=0}^n\big(x-\frac{k}{n}\big)^2{n\choose k}x^k(1-x)^{n-k}}\\}
\uncover<5->{&\leq2M\sqrt n\,\textcolor{blue}{\frac{x(1-x)}{n}}}
\uncover<6->{\leq\frac{M}{2\sqrt n}}
\uncover<7->{\leq \epsilon}
\end{align*}
\end{frame}


\begin{frame}{Demostraci\'on del Teorema de Bernstein}
En limpio, para $x\in\intunit$ y dada $\epsilon>0$, si $n$ es suficientemente grande tenemos que\pause
$$\big|f(x)-B_n(x)\big|\leq\left[\sum_1+\sum_2\right]\left|f(x)-f\big(\frac{k}{n}\big)\right|{n\choose k}x^k(1-x)^{n-k}<\epsilon+\epsilon.\pause$$
As\'\i, dada $\epsilon>0$ se eligi\'o $N\geq \max\left\{\frac{1}{\delta^4} \,\,,\,\frac{M^2}{\epsilon^2}\right\}$, de manera que si $n\geq N$ entonces
$$\big|f(x)-B_n(x)\big|<2\epsilon\qquad\text{para cualquier }\,x\in\intunit.\pause$$
Esto significa que $\|f-B_n\|_{\infty,\intunit}\to0$ si $n\to\infty$\hfill$\blacksquare$
\end{frame}





%%%%%%%%%%%%%%%%%%%%%%%%%%%%%%%%%%%%%%%%%%%%%%%%%%%%%%%%%%%
%%%%%%%%%%%%%%%%%%%%%%%%%%%%%%%%%%%%%%%%%%%%%%%%%%%%%%%%%%%
%%%%%%%%%%%%%%%%%%%%%\end{document}%%%%%%%%%%%%%%%%%%%%%%%%%%%%%%
%%%%%%%%%%%%%%%%%%%%%%%%%%%%%%%%%%%%%%%%%%%%%%%%%%%%%%%%%%%
%%%%%%%%%%%%%%%%%%%%%%%%%%%%%%%%%%%%%%%%%%%%%%%%%%%%%%%%%%%




\begin{frame}{T\'opicos a cubrir}

\begin{itemize}
\item Teorema de extensi\'on de Tietze\pause
\item M\'as sobre funciones continuas lineales por pedazos
\end{itemize}

El primer teorema provee condiciones que implican que una funci\'on continua en un subconjunto de $\rzp$ pueda extenderse como una funci\'on continua a todo $\rzp$.
\vskip5pt\pause
En el segundo punto revisaremos una caracerizaci\'on de funciones continuas lineales por pedazos por medio de combinaciones del valor abosluto.
\end{frame}


\begin{frame}
TEOREMA DE EXTENSI\'ON DE TIETZE
\end{frame}


\begin{frame}{Un ejemplo b\'asico de teorema de extensi\'on -- 1}
En la primera demostraci\'on que dimos del Teorema de Weierstrass, bajo la hip\'otesis de que $f:[0,1]\to\rz$ era continua, y $f(0)=f(1)=0$ extendimos la definici\'on de $f$ a todo $\rz$ como 
$$F(x)=\left\{\begin{array}{ll}
			f(x)&\text{si }\,x\in[0,1]\\
			0&\text{si }\,x\not\in[0,1]
		\end{array}\right.\pause$$
y afirmamos que $F$ era uniformemente continua en $\rz$.
\pause\vskip5pt
En efecto, dada $\epsilon>0$ y $x,y\in\rz$ tenemos varios casos:
\begin{itemize}
\item Si $x,y\not\in[0,1]$ entonces $F(x)=F(y)=0$ y no hay nada qué probar\pause
\item Si $x,y\in[0,1]$ entonces se sabe de la existencia de $\delta_1>0$ tal que $|x-y|<\delta$ implica $|f(x)-f(y)|<\epsilon$.\pause
\item Si $x\in [0,1]$ y $y\not\in[0,1]$ entonces $|f(x)-f(y)|=|f(x)|=|f(x)-f(1)|$ y de nuevo basta tomar la $\delta>0$ de la continuidad uniforme en $[0,1]$.
\end{itemize}
\end{frame}


\begin{frame}{Un ejemplo b\'asico de un teorema de extensi\'on -- 2}

\pause 
Sean $A,B\subset\rzp$ conjuntos cerrados y disjuntos.
\vskip4pt\pause
Sup\'ongase que tenemos $\varphi:A\cup B\to\rz$ definida como
$$\varphi(x)=\left\{\begin{array}{ll}
					0&\text{si }x\in A\\
					1&\text{si }x\in B
					\end{array}\right.\pause$$
y se quiere ``extender'' a esta funci\'on para tener una $\tilde\varphi:\rzp\to\rz$ de manera que sea continua en todo $\rzp$, cumpliendo adem\'as 
$$\tilde\varphi(x)=\varphi(x)\quad\text{si }x\in A\cup B,\pause\qquad 0\leq\tilde\varphi(x)\leq1\quad \text{para toda }x\in\rzp\pause$$ 
Definamos $d(x,A)=\inf\big\{\|x-y\|:y\in A\big\}$, \pause $d(x,B)=\inf\big\{\|x-y\|:y\in B\big\}$ \pause y 
$$\tilde\varphi(x)=\frac{d(x,A)}{d(x,A)+d(x,B)}\pause$$
es la soluci\'on al problema planteado.

\end{frame}



\begin{frame}{Teorema de extensi\'on de Tietze}

\begin{teo}
Sea $f:D\subset\rzp\to\rz$ continua y acotada, con $D\subset\rzp$ cerrado. \pause Entonces existe $g:\rzp\to\rz$ tal que $g(x)=f(x)$ para $x\in D$, y tal que $\|g\|_{\infty}=\|f\|_{\infty,D}$ \pause es decir
$$\sup\big\{|g(x)|:x\in\rzp\big\}=\sup\big\{|f(x)|:x\in D\big\}$$
\end{teo}
\pause
Como $f$ es acotada podemos definir $M=\sup\big\{|f(x)|:x\in D\big\}$, \pause y considerar 
$$A_1=\big\{x\in D:f(x)\leq-M/3\big\},\quad B_1=\big\{x\in D:f(x)\geq M/3\big\}\pause$$
N\'otese que $A_1$ y $B_1$ son cerrados en $\rzp$, por los teoremas de continuidad.

\end{frame}



\begin{frame}{Demostraci\'on del Teorema de Tietze}
Definamos
$$\tilde f(x)=\left\{\begin{array}{cl}
					-M/3&\text{si }x\in A_1\\
					M/3&\text{si }x\in B_1
					\end{array}\right.\pause$$
\begin{figure}[ht]
    \centering
    % \incfig{Extension0}
    \caption{Aspecto general de $f$ y los conjuntos $A_1$ y $B_1$}
\end{figure}

\end{frame}


\begin{frame}{Demostraci\'on del Teorema de Tietze}
Definiendo $\displaystyle\tilde\varphi_1(x)=\frac{3}{2M}\left(\tilde f(x)+M/3\right)$ tenemos que 
$$\tilde\varphi_1(x)=\left\{\begin{array}{ll}
					0&\text{si }x\in A_1\\
					1&\text{si }x\in B_1
					\end{array}\right.\qquad 0\leq\tilde\varphi_1(x)\leq1\pause$$
\begin{figure}[ht]
    \centering
    % \incfig{Extension1}
    \caption{Esquema de la gr\'afica de $\tilde\varphi_1$}
\end{figure}

\end{frame}


\begin{frame}{Demostraci\'on del Teorema de Tietze}
De acuerdo al Ejemplo B\'asico anterior existe $\psi_1:\rzp\to\rz$ que extiende a $\tilde\varphi_1$
\begin{figure}[ht]
    \centering
    % \incfig{Extension2}
    \caption{Esquema de la gr\'afica de $\varphi_1$}
\end{figure}
\pause
Sea $\displaystyle\varphi_1(x)=\frac{2M}{3}\psi_1(x)-\frac{M}{3}$, definida para toda $x\in\rzp$.
\end{frame}


\begin{frame}{Demostraci\'on del Teorema de Tietze}
N\'otese que se cumple
$$\varphi_1(x)=\left\{\begin{array}{cl}
					-M/3&\text{si }x\in A_1\\
					M/3&\text{si }x\in B_1
					\end{array}\right.\qquad \frac{-M}{3}\leq\varphi_1(x)\leq\frac{M}{3}$$
\vskip4pt\pause
Sea ahora $f_2=f-\varphi_1$. \pause\vskip5pt Entonces $f_2$ es continua en $D$ y $\displaystyle\sup\big\{|f_2(x)|:x\in D\big\}\leq \frac{2M}{3}$ porque:\pause\vskip5pt
\begin{itemize}
\item Para $x\not\in A_1\cup B_1$ sabemos $|f(x)|\leq M/3$ y $|\varphi_1(x)|\leq M/3$;\pause
\item Para $x\in A_1\cup B_1$, digamos $x\in B_1$, tenemos que 
$$\varphi_1(x)=M/3,\pause\quad \text{por lo que }\,|f(x)-\varphi_1(x)|=f(x)-M/3\leq 2M/3;$$ \pause Un argumento similar funciona para el caso $x\in A_1$:
$$\varphi_1(x)=-M/3,\quad\pause |f(x)-\varphi_1(x)|=-M/3-f(x)\leq -M/3+M=2M/3$$
\end{itemize}
\end{frame}


\begin{frame}{Demostraci\'on del Teorema de Tietze}
Si repetimos el procedimiento anterior con $f_2$, tenemos ahora una funci\'on con ``estatura'' igual a $2M/3$. 
\vskip5pt\pause
Al tomarle $1/3$ a esta cantidad tendremos que considerar truncaciones a la altura $\displaystyle\frac{M}{3}\left(\frac{2}{3}\right)$. \pause
Definimos entonces
$$A_2=\left\{x\in D:f(x)\leq-\frac{M}{3}\left(\frac{2}{3}\right)\right\},\quad B_2=\left\{x\in D:f(x)\geq \frac{M}{3}\left(\frac{2}{3}\right)\right\}\pause$$
COn el procedimiento anterior obtenemos ahora una funci\'on $\varphi_2:\rzp\to\rz$ tal que
$$\varphi_2(x)=\left\{\begin{array}{cl}
					-(M/3)(2/3)&\text{si }x\in A_2\\
					(M/3)(2/3)&\text{si }x\in B_2
					\end{array}\right.\qquad |\varphi_2(x)|\leq \frac{M}{3}\left(\frac{2}{3}\right)$$
\end{frame}


\begin{frame}{Demostraci\'on del Teorema de Tietze}
Si definimos $\textcolor{blue}{f_3}=f_2-\varphi_2\pause=f-\varphi_1-\varphi_2\pause\textcolor{blue}{=f-(\varphi_1+\varphi_2)}$, \pause se observa que $f_3$ es continua en $D$ y que $$\sup\big\{\textcolor{blue}{|f_3(x)|}:x\in D\big\}\leq \textcolor{red}{\left(\frac{2}{3}\right)^2M}\pause=2\left(\frac{2}{3}\frac{M}{3}\right).$$
Esto es cierto porque:\pause\vskip5pt
\begin{itemize}
\item Para $x\not\in A_2\cup B_2$ sabemos $|f_2(x)|\leq (M/3)(2/3)$ y $|\varphi_2(x)|\leq (M/3)(2/3)$;\pause
\item Para $x\in A_2\cup B_2$, digamos $x\in B_2$, tenemos que $\varphi_2(x)=(M/3)(2/3)$ y \pause
\end{itemize}\vskip4pt
$$|f_2(x)-\varphi_2(x)|=f_2(x)-\left(\frac{M}{3}\right)\left(\frac{2}{3}\right)\leq 2\left(\frac{M}{3}\right)-\left(\frac{M}{3}\right)\left(\frac{2}{3}\right)=2\left(\frac{2}{3}\frac{M}{3}\right);$$ \pause Un argumento similar funciona para el caso $x\in A_1$, pues en este caso $\varphi_2(x)=-M/3$ y 
$$|f_2(x)-\varphi_2(x)|=-\left(\frac{M}{3}\right)\left(\frac{2}{3}\right)-f_2(x)\leq -\left(\frac{M}{3}\right)\left(\frac{2}{3}\right)+2\left(\frac{M}{3}\right)=2\left(\frac{2}{3}\frac{M}{3}\right)$$

\end{frame}


\begin{frame}{Demostraci\'on del Teorema de Tietze}

Con la construcci\'on anterior repetida recursivamente obtenemos una sucesi\'on $\big(\varphi_n\big)$ definidas en $\rzp$ cumpliendo\pause
\begin{align}
\textcolor{blue}{\left|f(x)-\big(\varphi_1(x)+\cdots+\varphi_n(x)\big)\right|}\leq\textcolor{red}{\left(\frac{2}{3}\right)^nM}\qquad\text{para toda }x\in D,\tag{*}\\
|\varphi_n(x)|\leq \frac{M}{3}\left(\frac{2}{3}\right)^{n-1}\hskip2cm\text{para toda }x\in \rzp.\tag{**}
\end{align}\pause
Definimos la funci\'on continua $g_n=\varphi_1+\varphi_2+\cdots+\varphi_n$. \pause Entonces para $m\geq n$ y $x\in \rzp$ tenemos por (**)
$$|g_n(x)-g_m(x)|=|\varphi_{n+1}(x)+\cdots+\varphi_m(x)|\pause\leq\frac{M}{3}\left(\frac{2}{3}\right)^n\left[1+\frac{2}{3}+\left(\frac{2}{3}\right)^2+\cdots\right]\pause$$
En conclusi\'on $\displaystyle |g_n(x)-g_m(x)|\leq\left(\frac{2}{3}\right)^nM$ para toda $x\in \rzp$.

\end{frame}



\begin{frame}{Demostraci\'on del Teorema de Tietze - 4}

De lo anterior concluimos que $(g_n)$ es de Cauchy uniformemente, y uniformemente acotada, por lo que existe $g$ continua y acotada en todo $\rzp$ que es el l\'\i mite uniforme de la sucesi\'on $(g_n)$.
\vskip4pt\pause
Como cada $g_n$ es continua y acotada en $\rzp$ entonces $g$ es continua en todo $\rzp$.
\pause\vskip4pt
Por la condici\'on $(*)$ tenemos adem\'as que $|f(x)-g_n(x)|\leq\left(\frac{2}{3}\right)^nM$ para $x\in D$.
\pause\vskip4pt
Por unicidad de l\'\i mites $f(x)\equiv g(x)$ para toda $x\in D$, y entonces $\|f\|_{\infty,D}=\|g\|_{\infty,D}$.
\pause\vskip4pt
Finalmente por la condici\'on (**) tenemos que para $x\in \rzp$
$$|g_n(x)|\leq \frac{M}{3}\left[1+\frac{2}{3}+\left(\frac{2}{3}\right)^2+\cdots+\left(\frac{2}{3}\right)^{n-1}\right]< M\pause$$
lo cual implica que $|g(x)|\leq M$ para $x\in\rzp$. \pause En conclusi\'on $\|g\|_{\infty}\leq M$.
\pause\vskip4pt
Como adem\'as es claro que $\|g\|_{\infty}\geq M$ entonces concluimos $\|g\|_{\infty}= M$\hfill$\blacksquare$
\end{frame}



\begin{frame}

FUNCIONES CONTINUAS LINEALES POR PEDAZOS

\end{frame}


\begin{frame}{Translaciones de la funci\'on {\em Valor Absoluto}}
Definamos 
$$\abs(x)=|x|,\qquad\abs_a(x)=|x-a|\qquad\text{para }\,x\in\intunit,\,a\in\rz.\pause$$

\begin{figure}
\includegraphics[ height=4cm, width=4cm]{ValoresAbsolutos.png}
\caption{Gr\'aficas de $y=\abs(x-a)$  con $a=0, 3, -1/2$}
\end{figure}

\end{frame}


\begin{frame}{El resultado principal}
El resultado que queremos establecer es el siguiente:

\begin{teo}
Si $f:\intunit\to\rz$ es continua y lineal por pedazos, entonces $f$ es combinaci\'on lineal de funciones $\abs_a$ para ciertas $a\in\intunit$.\pause
\end{teo}

Por hip\'otesis existen $0=x_0<x_1<\cdots<x_n=1$ de manera que $f$ es lineal en cada subintervalo $I_k=[x_{k-1},x_k]$, $k=1,\dots,n$.
\vskip5pt\pause
Definamos $S=\gene\big\{\abs_{x_k}:0\leq k\leq n\big\}$. \pause Entonces $S$ es un subespacio vectorial de $C[0,1]$
\vskip5pt\pause
Incidentalmente, n\'otese que las funciones constantes pertenecen a $S$ pues para $x\in\intunit$ tenemos
$$\abs_{x_0}(x)-\abs_{x_n}(x)=x+(1-x)=1$$
\end{frame}


\begin{frame}{Demostraci\'on del teorema}\pause
Haremos tres modificaciones a la funci\'on $\abs_{x_k}(x)$.
\pause\vskip5pt
Definamos 
$$R_k(x)=\frac{1}{2}\big(\abs_{x_k}(x)+(x-x_k)\big)\pause=\left\{\begin{array}{cc}
													0&\text{si }x\leq x_k\\
													x-x_k&\text{si }x> x_k
												\end{array}\right.\pause$$
\begin{figure}
\includegraphics[ height=4cm, width=4cm]{Rk.png}
\caption{Gr\'afica de $R_{k}$  con $x_k=0.2$}
\end{figure}
\end{frame}


\begin{frame}{Demostraci\'on del teorema}
Ahora definimos 
$$J_k=\frac{R_k-R_{k-1}}{x_k-x_{k-1}}\pause\quad\text{ y notamos que }\,\,J_k(x_j)=\left\{\begin{array}{cc}
													0&\text{si }j< k\\
													1&\text{si }j\geq k
												\end{array}\right.\pause$$
\begin{figure}
\includegraphics[ height=4cm, width=4cm]{Jk.png}
\caption{Gr\'afica de $J_{k}$  con $x_k=0.2$, suponiendo que $x_{k-1}=0.1$}
\end{figure}
\end{frame}


\begin{frame}{Demostraci\'on del teorema}
Finalmente definimos 
$$H_k=J_k-J_{k+1}\pause\quad\text{ y notamos que }\,\,H_k(x_j)=\left\{\begin{array}{cc}
													1&\text{si }k=j\\
													0&\text{si }j\neq k
												\end{array}\right.\pause$$
\begin{figure}
\includegraphics[ height=4cm, width=4cm]{Hk.png}
\caption{Gr\'afica de $H_{k}$  con $x_k=0.2$, suponiendo que $x_{k-1}=0.1$ y $x_{k+1}=0.3$}
\end{figure}
\end{frame}


\begin{frame}{Demostraci\'on del teorema}
Obs\'ervese que $R_k,J_k,H_k\in S$, \pause y adem\'as
$$\sum_{k=0}^nf(x_k)H_k$$
es una funci\'on lineal por pedazos que coincide con $f$ en todo $x_k$.
\vskip5pt\pause
Debe entonces ocurrir que 
$$f(x)=\sum_{k=0}^nf(x_k)H_k(x)\qquad\text{para todo }\,x\in\intunit.\pause$$
Por tanto $f\in S$\hfill$\blacksquare$
\end{frame}







\end{document}




















