\documentclass[a4paper]{article}
\usepackage{paquetePreset}
\usepackage{amssymb}
\usepackage{amsmath}
\usepackage[margin=3cm]{geometry} 
\usepackage{mathtools}
\usepackage{xcolor}
% \usepackage{quattrocento}
\usepackage[shortlabels]{enumitem}
\usepackage{extarrows}
\usepackage[usestackEOL]{stackengine}

\newcommand{\dif}[1]{\;\mathrm{d}#1}

\begin{document}
\begin{raggedleft}
    \Large
    \textsc{Métodos de resolución}
\end{raggedleft}

\section{Ecuaciones separables}
\section{Ecuaciones exactas}
Ecuación de la forma 

\begin{equation}
    \label{eq:forma-exacta}
    \underbrace{{\color{red}M}(x,y) \dif{{\color{blue}x}}}_{A} + \underbrace{{\color{blue}N}(x,y) \dif{{\color{red} y}}}_{B} = 0  
\end{equation}
que cumple:
\begin{equation*}
    \frac{\partial {\color{red}M}(x,y)}{\partial {\color{red}y }} = \frac{\partial {\color{blue} N}(x,y)}{\partial {\color{blue} x}}.
\end{equation*}

\begin{obs}
    $M$ y $N$ se derivan con respecto a la variable que \textbf{no} está en el diferencial que tienen al lado.
    Es decir, van ``cruzadas''.
    Por ejemplo $\dif{x} \mapsto \partial y$.
\end{obs}

Recordamos que $\displaystyle \frac{\partial F(x,y)}{\partial x} = M(x,y)$, y $\displaystyle \frac{\partial F(x,y)}{\partial y} = N(x,y)$

\begin{enumerate}
    \item Elegimos $M$, o $N$ para integrar dependiendo de cual sea más fácil, e integramos con respecto a la variable del diferencial que tienen en la ec. \eqref{eq:forma-exacta}.
    
    Ahora hacemos la integración, agregando una función $g$ de la variable que no estamos integrando (en este caso ${\color{red} y}$)
    \begin{align*}
        F(x,y) &= \int \underbrace{M(x,y) \dif{{\color{blue}x}}}_{A} + g({\color{red}y}) \\
        &= X(x,y) + g({\color{red}y}) \tag{$\star$}
    \end{align*}
    La expresión ($\star$) nos da la solución general más tarde.

    \item Ahora derivamos $F$ con respecto a la variable que no se integró en el paso anterior (en este caso ${\color{red} y}$), para después igualar a la función que no se integró en el paso anterior (en este caso, $N$).
    \begin{align*}
        \frac{\partial F(x,y)}{\partial y} &= \frac{\partial X(x,y)}{\partial y} + g'(y) = N(x,y) 
    \end{align*}

    \item Despejamos $g'$ de la ecuación obtenida en el paso anterior, y la integramos para obtener $g$.
    \begin{align*}
        g'(y) &= N(x,y) - \frac{\partial X(x,y)}{\partial y} \\
        g(y) &= \int \left[ N(x,y) - \frac{\partial X(x,y)}{\partial y}  \right] \dif{y} \\
    \end{align*}

    \item 
    Tomando la ecuación ($\star$), la solución general es
    \begin{align*}
        (\star) &= c \\
        X(x,y) + g(y) &= c
    \end{align*}
    Con $c$ una constante, $X(x,y)$ obtenida en paso 1, y $g(y)$ obtenida en paso 3.
\end{enumerate}


\section{Ecuaciones con factor integrante}
Ecuación de la forma 

\begin{equation}
    \label{eq:forma-exacta-2}
    \underbrace{{\color{red}M}(x,y) \dif{{\color{blue}x}}}_{A} + \underbrace{{\color{blue}N}(x,y) \dif{{\color{red} y}}}_{B} = 0  
\end{equation}
que cumple:
\begin{equation*}
    \frac{\partial {\color{red}M}(x,y)}{\partial {\color{red}y }} \neq \frac{\partial {\color{blue} N}(x,y)}{\partial {\color{blue} x}}.
\end{equation*}

Definimos
\begin{align*}
    p(x) &= \frac{\frac{\partial M}{\partial y} - \frac{\partial N}{\partial x}}{N} &\text{\&} && p(y) = \frac{\frac{\partial N}{\partial x} - \frac{\partial M}{\partial y}}{M}
\end{align*}

\begin{obs}
    Una forma fácil de recordar la forma de la función $p$.
    La empezamos a llenar por el lado ``derecho'' y el denominador.
    \begin{equation*}
        p({\color{red} \diamond}) = \frac{\phantom{\frac{\partial M}{\partial x}} - \frac{\partial {\color{green} \spadesuit}}{\partial{\color{red} \diamond}}}{{\color{green} \spadesuit}}
    \end{equation*}
    Notamos un patrón. En el numerador, el término que se resta ($\frac{\partial {\color{green} \spadesuit}}{\partial {\color{red} \diamond} }$) tiene a la variable de la función abajo, y el denominador de $p$ arriba.
\end{obs}

\begin{enumerate}
    \item Identificar $p(x)$ o $p(y)$ dependiendo de cual sea una función univariada, o cual sea más manejable.
    Supongamos que trabajamos con $p(x)$.

    \item Calcular el factor integrante como
    \begin{equation*}
        \mu(x) = \exp \left( \int p(x) \dif{x} \right)
    \end{equation*}

    \item Multiplicar la ec. (2) en ambos lados por $\mu(x)$ para obtener 
    \begin{equation}
        \label{eq:exactificada}
        \mu(x) M(x,y) \dif{x} + \mu(x) N(x,y) \dif{y} = 0
    \end{equation}
    La ecuación \ref{eq:exactificada} ya es de la forma exacta, y puede ser resuelta como en la \S 2.
\end{enumerate}

\section{Ecuaciones lineales}
Ecuación de la forma 
\begin{equation}
    \label{eq:lineal}
    \frac{\dif{y}}{\dif{x}} + p(x) \cdot y = q(x).
\end{equation}

\begin{enumerate}
    \item Calculamos el factor integrante como 
    \begin{equation*}
        \mu(x) = \exp \left( \int p(x) \dif{x} \right).
    \end{equation*}

    \item Multiplicamos ambos lados de \eqref{eq:lineal} por $\mu(x)$, de tal forma que podemos escribir
    \begin{equation*}
        \frac{\mathrm{d}}{\dif{x}} \left( y \cdot \mu \right) = \mu(x) \cdot q(x).
    \end{equation*}

    \item Integramos de ambos lados para obtener
    \begin{equation*}
        y \cdot \mu(x) = \int \mu(x) \cdot q(x) \dif{x} + c.
    \end{equation*}

    \item La solución general en forma estándar es
    \begin{equation*}
        y = \frac{\int \mu(x) q(x) \dif{x} + c }{\mu(x)}.
    \end{equation*}
\end{enumerate}

\section{Ecuaciones de Bernoulli}

\section{Ecuaciones de Riccati}
Ecuación de la forma 
\begin{equation}
    \label{eq:riccati}
    \frac{\dif{y}}{\dif{x}} = p(x) + q(x)y + r(x)y 2
\end{equation}
se convierte en lineal con el cambio de variable $z = (y - y_p)^{-1}$ donde $y_{p}$ es una solución particular.

\begin{enumerate}
    \item Del cambio de variable obtenemos
    \begin{equation*}
        y = z^{-1} + y_{p}
    \end{equation*}

    \item Calculamos el lado izquierdo de la ec. \eqref{eq:riccati}.
    \begin{equation*}
        \frac{\dif{y}}{\dif{x}} = \frac{\mathrm{d}}{\dif{x}} \left( z^{-1} + y_p \right) = -z^{-2} + \frac{\dif{z}}{\dif{x}} 
    \end{equation*}

    \item Calculamos el lado derecho de la ec. original sustituyendo $y$ con $z^{-1} + y_p$.
    Llamamos a el lado derecho con el cambio de variable:
    \begin{align*}
       &= p(x) + q(x) (z^{-1} - y_p) + r(x) (z^{-1} - y_p)^{2} \\
       &= p(x) + z^{-1} q(x) - y_p q(x) + z^{-2} r(x) + y^{2}_{p} r(x) - 2 z^{-1} y_p r(x)
    \end{align*}

    \item Juntamos ambos lados, y despejamos $\frac{\dif{x}}{\dif{y}}$.
    \begin{align*}
        z^{-2}  \frac{\dif{z}}{\dif{x}} &= p(x) + z^{-1} q(x) - y_p q(x) + z^{-2} r(x) + y^{2}_{p} r(x) - 2 z^{-1} y_p r(x) \\
        \frac{\dif{z}}{\dif{x}} &= z^{2} p(x) + z q(x) - y_p z^{2} q(x) + r(x) + y^{2}_{p} z^{2} r(x) - 2 z y_p r(x)
    \end{align*}
    La ecuación resultante es lineal 
\end{enumerate}

\end{document}